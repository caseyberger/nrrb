\documentclass[../RotatingBosons.tex]{subfiles}
\begin{document}
\section{Updates to make in the notes}
\subsection{TO DO}
\begin{itemize}
	\item Second complexify the drift function
	\item Double check the derivation of the observables (in Appendix~\ref{NRRBObservables}) -- this is done in your handwritten notes, not updated to the latex doc yet
	\item Second complexify the observables
\end{itemize}

\subsection{DONE}
\begin{itemize}
	\item Double check the discretization of $S_{\tau, r}  =  \phi^{*}_{r}\phi_{r}- \phi^{*}_{r}e^{d \tau \mu} \phi_{r - \hat{\tau}} $
	\item Double check the discretization of $S_{\del, r}  =  \frac{d}{m} \phi^{*}_{r}\phi_{r} - \frac{1}{2m}\sum_{i = \pm x,y} \phi^{*}_{r}\phi_{r+\hat{i}}$
	\item Double check the discretization of $S_{\mathrm{tr},r} = \frac{m}{2}\omega_{\text{trap}}^{2}(x^{2}+y^{2})\phi_{r}^{*}\phi_{r-\hat{\tau}}$
	\item Double check the discretization of $S_{\omega, r} =  i \omega_{z} \left[ (x-y)\phi_{r}^{*}  \phi_{r-\hat{\tau}} - \widetilde{x} \phi_{r}^{*} \phi_{r-\hat{y}-\hat{\tau}} + \widetilde{y} \phi_{r}^{*}  \phi_{r-\hat{x}-\hat{\tau}} \right] $
	\item Double check the discretization of $S_{\text{int}, r} = \lambda \left(\phi_{r}^{*} \phi_{r-\hat{\tau}}\right)^{2}$	
	\item Copy derivation from GoodNotes -- discretization updated (the version in this doc is correct, the new one is just more clear where the factors of $a$ go, which don't matter for now because we've set $a = 1$)
	\item check signs in Appendix\ref{NRAction}
	\item Double check the math when you write $\phi = \phi_{1} + i\phi_{2}$ (in Appendix~\ref{FirstComplexification})
	\item Determine a more natural way to write the interaction term in Appendix~\ref{FirstComplexification}
	\item Update notes to be more explicit about multidimensional cases... We're using $d=2$, but we want to generalize this. Is the best way to write it generally? Or to derive it in all three dimensions seperately?
	\item Double check the derivation of the drift functions (in Appendix~\ref{NRRBCLM}) 
\end{itemize}

\section{General Notes and Reminders}
*note that all interaction in this system is entirely governed by $\lambda$
\subsection{Stuff to look into}
\begin{itemize}
	\item From Ref.~\cite{1903.08672} (see master lit review for citation):

``A balance between the kinetic energy  {\bf [see ref in lit review]} and the interaction energy  {\bf [see ref in lit review]}, viz. Eq. (10), of a BEC leads to a typical length scale called coherence length {\bf [see ref in lit review]} for a weakly interacting BEC [11]. This
quantity is relevant for super-fluid effects. For instance, it provides the typical size of the core of quantized vortices [7, 11, 13]. "

	\item From ref.~\cite{2019arXiv191011461E} (see master lit review for citation):
	
``The vortex nucleation process in a rotating scalar Bose- Einstein condensate provides a unique model system to study the emergence of a quantum phase transition from the microscopic few-body regime to the thermodynamic limit. Even in the presence of interactions, the nucleation process of the first vortex is associated with exact linearity of the ground state energy as a function of angular momentum which leads to a discontinuous transition between the non-rotating ground state and the unit vortex."

\end{itemize}

\subsection{Ideas}
Might it be feasible to program a GPU for the CL evolution? It could then simultaneously update the lattice very quickly.

Other methods:
\begin{itemize}
	\item Path Optimization Method (Ohnishi et al)
	\item Tensor Network Systems (Banuls et al)
	\item World Line Methods??
\end{itemize}

\subsection{Goals:}
\begin{enumerate}
    \item Reproduce results of Gert Aarts paper - COMPLETED
    \item Reproduce the free gas - COMPLETED (Not for AMReX yet)
    \item Check against virial theorems for trapped bosons - IN PROGRESS
   
    I'm not sure that I can check the virial theorem here... how would I calculate kinetic energy only? Also we have rotation, interaction, and trap, not just trap.

    \item Compare with Hayata/Yamamoto paper - IN PROGRESS
    \item Some new directions to take it:
    \begin{itemize}
        \item Fixed values of angular momentum
        \item Polarized bosons
        \item Mass-imbalanced bosons
        \item Efimov effect and 3-body forces
        \item Some interesting problems in $2+1$ dimensions
        \item Virial coefficients for bosons at unitarity (maybe talk to Chris)
        \item Auxiliary field to represent interaction - will allow us to study phase transitions as a function of $N_{f}$ ***
        \item Can we ever find a way to vary the dimensionality continuously?
    \end{itemize}
\end{enumerate}

\end{document}