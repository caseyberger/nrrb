\documentclass[onecolumn, 12pt]{report}
\usepackage{hyperref,mathtools,amssymb}
\usepackage{cite}
\usepackage{graphicx}
\graphicspath{{./images/}}
\usepackage{subcaption}
\usepackage[toc,page]{appendix} 
\usepackage{subfiles}
\usepackage[margin=0.75in]{geometry}
%%%%%%%%%%%%%%%%%%%%%%%%%%%
%YOU MAY NEED THIS FOR ORGANIZATION: https://latex.org/forum/viewtopic.php?t=13267

%%%%%%%%%%%%%%%%%%%%%%%%%%%%%%%%%%%%%%%%%
%this allows you to place formatted code in your document
\usepackage{listings}
\usepackage{color}

\definecolor{dkgreen}{rgb}{0,0.6,0}
\definecolor{gray}{rgb}{0.5,0.5,0.5}
\definecolor{mauve}{rgb}{0.58,0,0.82}

\lstset{frame=tb,
  language=python,
  aboveskip=3mm,
  belowskip=3mm,
  showstringspaces=false,
  columns=flexible,
  basicstyle={\small\ttfamily},
  numbers=none,
  numberstyle=\tiny\color{gray},
  keywordstyle=\color{blue},
  commentstyle=\color{dkgreen},
  stringstyle=\color{mauve},
  breaklines=true,
  breakatwhitespace=true,
  tabsize=3
}

%You can change default language in the middle of document with \lstset{language=Java}
%Example of use:
%\begin{lstlisting}
%// Hello.java
%import javax.swing.JApplet;
%import java.awt.Graphics;
%
%public class Hello extends JApplet {
%   public void paintComponent(Graphics g) {
%        g.drawString("Hello, world!", 65, 95);
%    }    
%}
%\end{lstlisting}
%%%%%%%%%%%%%%%%%%%%%%%%%%%%%%%%%%%%%%%%%

\title{NonRelativistic Rotating Bosons via Complex Langevin}
%\subject{Many-Body Quantum Mechanics}
\author{Casey E. Berger}
\date{\today}


\newcommand{\etal}{{\it et al.}}
\newcommand{\beq}{\begin{equation}}
\newcommand{\eeq}{\end{equation}}
\newcommand{\bea}{\begin{eqnarray}}
\newcommand{\eea}{\end{eqnarray}}

\def\CP{{\mathcal P}}
\def\CC{{\mathcal C}}
\def\CH{{\mathcal H}}
\def\CW{{\mathcal W}}
\def\CO{{\mathcal O}}
\def\CZ{{\mathcal Z}}
\def\CD{{\mathcal D}}
\def\del{{\nabla}}

\newcommand*\dif{\mathop{}\!\mathrm{d}}

\newcommand{\note}[1]{{\color{red} \bf[NOTE: #1]}}

\begin{document}
\begin{titlepage}
\maketitle
\end{titlepage}
\tableofcontents

%%%%%%%%%% NOTES AND TO-DO LIST  %%%%%%%%%%%%%%%%%%%
%\chapter{Notes To Self and To-Do List}
%\subfile{./notes/notes_to_self.tex}

%%%%%%%%%  RECREATING THE AARTS PAPER  %%%%%%%%%%%%%%%
%\chapter{Warm-Up Project}
%\subfile{./aartspaper/aarts.tex}

%%%%%%%%%  NONRELATIVISTIC ROTATING BOSE GAS  VIA CL %%%%%%%
%\chapter{Motivation, Context, and History} %this should be a lit review. Cull some from your dissertation but read more papers about work on mean-field theory methods

\chapter{The Method: Complex Langevin}
\subfile{./nrrb/method/nrrbmethod.tex}

\chapter{The Code}
\subfile{./notes/code_instructions.tex}

\chapter{Results and Analysis}
\subfile{./nrrb/results/nrrbresults.tex}

%%%%  NONRELATIVISTIC ROTATING BOSE GAS  VIA OTHER METHODS %%%%
%\subfile{./3.alternatives/alternatives.tex}

%%%%%%%%%%   APPENDICES   %%%%%%%%%%%%%%%%%%%%%%%%

\begin{appendices}
%%%%%%%%%%%%%  AARTS PAPER  %%%%%%%%%%%%%%%%%%%%%
%\chapter{Warm Up Project Derivations}
%\subfile{./aartspaper/derivations/aartsderivations.tex}
%\chapter{Warm Up Project Code}
%\subfile{./aartspaper/code/aartscode.tex}
%\subfile{./aartspaper/analysis/aartspythonscripts.tex}

%%%%%%%%%%%%%%%  NRRB  %%%%%%%%%%%%%%%%%%%%%%%
%\chapter{NRRB Code}
%\subfile{./nrrb/code/nrrbcode.tex}
\chapter{NRRB Derivations}
\subfile{./nrrb/derivations/nrrb_action.tex}
\subfile{./nrrb/derivations/nrrb_drift_functions.tex}
\subfile{./nrrb/derivations/nrrb_observables.tex}
%\subfile{./nrrb/derivations/nrrb_analytical_solns}
%\subfile{./nrrb/analytical/freegas.tex}


\end{appendices}
\bibliography{nrrbrefs.bib}{}
%\bibliographystyle{ieeetr}
\bibliographystyle{plain}
\end{document}