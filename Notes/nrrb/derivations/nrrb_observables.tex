\documentclass[../../RotatingBosons.tex]{subfiles}

\begin{document}

%%%%%%%%%%%%%%%%%%%%%%%%%%%%%%%%%%%%%%%%%%%%%
%%%%%%%%%%%%%%%%%% OBSERVABLES %%%%%%%%%%%%%%%%%%
%%%%%%%%%%%%%%%%%%%%%%%%%%%%%%%%%%%%%%%%%%%%%

\section{\label{NRRBObservables}Lattice Observables}
The observables we are interested in calculating are:
\begin{itemize}
	\item particle density (local and average density)
	\item angular momentum
	\item field modulus squared
	\item harmonic trapping potential energy
	\item interaction potential energy
	\item kinetic energy
	\item total energy
	\item circulation
\end{itemize}
Most of our observables can be determined by taking appropriate derivatives of $\ln \CZ$, with the exception of the field modulus squared and the circulation. The general form for this argument is shown below. 

We have a partition function, $\CZ$, %which in its discretized form is a product over all field configurations, $\phi_{i}$:
%
%\beq
%\CZ = \prod_{\phi_{i}}e^{-S_{lat}[\phi_{i}]},
%\eeq
%
%whose log is a sum over all field configurations:
%
%\beq
%\ln \CZ = - \sum_{\phi_{i}} S_{lat}[\phi_{i}]
%\eeq
%
\beq
\CZ = \int \CD \phi e^{-S[\phi]}
\eeq
%
Observables are calculated via a path integral as well, where the integrand includes the observable as a function of the fields, weighted by $e^{-S}$:
%
\beq
\langle \CO \rangle =\frac{1}{\CZ} \int \CD \phi \CO[\phi]e^{-S[\phi]}
\eeq
%
If we take a derivative of the log of $\CZ$ with respect to some parameter $\alpha$, we get something that looks very similar to this expression:
%
\beq
\frac{\partial \ln \CZ}{\partial \alpha} = \frac{1}{\CZ}\frac{\partial \CZ}{\partial \alpha} = \frac{1}{\CZ}\int \CD \phi ( -\frac{\partial S[\phi,\alpha]}{\partial \alpha} ) e^{-S[\phi,\alpha]}
\eeq
%
which suggests that we can compute observables by taking derivatives of the action
%
\beq
\hat{\CO}_{\alpha} = - \frac{\partial}{\partial \alpha} S[\phi, \alpha]
\eeq
%
In our statistical simulation, we seek to compute average values of these observables, weighted by $e^{-S}$. The complex Langevin simulation ensures that our samples are distributed according to the proper weight, so we are able to perform a simple average of the observables after the simulation has thermalized.

\note{where does the beta come in? Or do we just need to take derivatives wrt our lattice parameters (e.g. $\bar{\mu}$ instead of $\mu$)?}

%%%%%%%%%%%%%%%%%%%%%%%%%%%%%%%%%%%%%%%%%%%%%
%%%%%%%%%%%%%%%%%%%%%%%%%%%%%%%%%%%%%%%%%%%%%
\subsection{Density}
To calculate the density, first we need to determine the total number of particles in the system. We can do this by taking a derivative of the lattice action with respect to the chemical potential. Since the only part of the lattice action that depends on $\bar{\mu}$ is $S_{\mu}$, we can ignore the rest of the action:
%
\bea
\hat{N}[\phi_{i}] &=& - \frac{\partial}{\partial \bar{\mu}} S_{lat} = - \frac{\partial}{\partial \bar{\mu}} S_{\mu}  = - \frac{\partial}{\partial \bar{\mu}} \sum_{\vec{x},\tau}a^{d}\left(\phi_{r}^{*}\phi_{r} -e^{\bar{\mu}}\phi_{r}^{*}\phi_{r - \hat{\tau}}\right) \nonumber \\
&=& \sum_{\vec{x},\tau}a^{d}\left(e^{\bar{\mu}}\phi_{r}^{*}\phi_{r - \hat{\tau}}\right).
\eea
%
This can be written in terms of $S_{\mu}$:
%
\beq
\hat{N}[\phi_{i}] =  \sum_{\vec{x},\tau}a^{d}\phi_{r}^{*}\phi_{r} -S_{\mu} 
\eeq
%
To determine the density, we simply normalize the particle number by the spatial lattice volume:
%
\beq
\hat{n}[\phi_{i}] =  \frac{1}{(a N_{x})^{d}} \hat{N}[\phi_{i}]
\eeq
%

\subsubsection{Complexification}
When we apply the complexification to the real fields, $\phi_{a/b}\rightarrow \phi_{a/b}^{R} + i \phi_{a/b}^{I}$, we get:
%
\bea
\hat{n} &=& -\frac{1}{2 N_{x}^{d}} e^{\bar{\mu}} \sum_{\vec{x},\tau}\sum_{a,b=1}^{2} \left( \delta_{ab} \phi_{a,r}\phi_{b,r-\hat{\tau}} + i \epsilon_{ab} \phi_{a,r}\phi_{b,r-\hat{\tau}}\right) \nonumber\\
%
&=& -\frac{1}{2N_{x}^{d}} e^{\bar{\mu}} \sum_{\vec{x},\tau}\sum_{a,b=1}^{2}\left[
\delta_{ab}(\phi_{a,r}^{R} \phi_{r,b-\hat{\tau}}^{R} - \phi_{a,r}^{I} \phi_{r,b-\hat{\tau}}^{I} ) 
- \epsilon_{ab}(\phi_{a,r}^{R} \phi_{r,b-\hat{\tau}}^{I} +  \phi_{a,r}^{I} \phi_{r,b-\hat{\tau}}^{R} )\right] \nonumber\\
%
&& -\frac{i}{2N_{x}^{d}} e^{\bar{\mu}} \sum_{\vec{x},\tau}\sum_{a,b=1}^{2}\left[
\delta_{ab}(\phi_{a,r}^{R} \phi_{r,b-\hat{\tau}}^{I} + \phi_{a,r}^{I} \phi_{r,b-\hat{\tau}}^{R} )
+ \epsilon_{ab}(\phi_{a,r}^{R} \phi_{r,b-\hat{\tau}}^{R} - \phi_{a,r}^{I} \phi_{r,b-\hat{\tau}}^{I} )
\right]
%& = & -\frac{1}{2} e^{\bar{\mu}} \sum_{r}\sum_{a,b=1}^{2} \left[ \delta_{ab} \left( \phi_{a,r}^{R}\phi_{b,r-\hat{\tau}}^{R} + i \phi_{a,r}^{R}\phi_{b,r-\hat{\tau}}^{I} + i \phi_{a,r}^{I}\phi_{b,r-\hat{\tau}}^{R} - \phi_{a,r}^{I}\phi_{b,r-\hat{\tau}}^{I} \right) \right. \nonumber \\
%&&\left. + \epsilon_{ab} \left(i \phi_{a,r}^{R}\phi_{b,r-\hat{\tau}}^{R} - \phi_{a,r}^{R}\phi_{b,r-\hat{\tau}}^{I} - \phi_{a,r}^{I}\phi_{b,r-\hat{\tau}}^{R} - i \phi_{a,r}^{I}\phi_{b,r-\hat{\tau}}^{I} \right)\right] \nonumber \\
%& = & -\frac{1}{2} e^{\bar{\mu}} \sum_{r}\sum_{a,b=1}^{2} \left[ \delta_{ab} \left( \phi_{a,r}^{R}\phi_{b,r-\hat{\tau}}^{R} - \phi_{a,r}^{I}\phi_{b,r-\hat{\tau}}^{I} \right) - \epsilon_{ab} \left( \phi_{a,r}^{R}\phi_{b,r-\hat{\tau}}^{I} + \phi_{a,r}^{I}\phi_{b,r-\hat{\tau}}^{R} \right)\right] \nonumber \\
%& & - \frac{i}{2} e^{\bar{\mu}} \sum_{r}\sum_{a,b=1}^{2} \left[  \delta_{ab} \left( \phi_{a,r}^{R}\phi_{b,r-\hat{\tau}}^{I} +  \phi_{a,r}^{I}\phi_{b,r-\hat{\tau}}^{R}\right) + \epsilon_{ab} \left( \phi_{a,r}^{R}\phi_{b,r-\hat{\tau}}^{R}- i \phi_{a,r}^{I}\phi_{b,r-\hat{\tau}}^{I}\right)\right]
\eea
%
This was done in Mathematica.


%%%%%%%%%%%%%%%%%%%%%%%%%%%%%%%%%%%%%%%%%%%%%
%%%%%%%%%%%%%%%%%%%%%%%%%%%%%%%%%%%%%%%%%%%%%
\subsection{Local Density}
The local density is simply the density function without the sum over the lattice. Since we are concerned with spatial density, we can either average our results over the extent of the time lattice
%
\beq
\hat{n}_{r}[\phi_{i},\vec{x}] =  \frac{1}{N_{\tau}(a N_{x})^{d}} \sum_{\tau} a^{d}\left(e^{\bar{\mu}}\phi_{r}^{*}\phi_{r - \hat{\tau}}\right)
\eeq
%
or take the density profile of a single time slice:
%
\beq
\hat{n}_{r}[\phi_{i},\vec{x},\tau] =  \frac{1}{(a N_{x})^{d}} a^{d}\left(e^{\bar{\mu}}\phi_{r}^{*}\phi_{r - \hat{\tau}}\right)
\eeq
%
Averaging over the extent of the time lattice should produce better results (i.e. we will have a mean and a standard error, which gives us a statistical value to evaluate), so we will do this.

\subsubsection{Complexification}
The difference between this observable and the density is simply that we evaluate the local density at each site and then do not average over the entire spatial lattice. The complexified observable value is therefore the same, after removing the sum over $\vec{x}$.

%%%%%%%%%%%%%%%%%%%%%%%%%%%%%%%%%%%%%%%%%%%%%
%%%%%%%%%%%%%%%%%%%%%%%%%%%%%%%%%%%%%%%%%%%%%
\subsection{Field Modulus Squared}
The field modulus squared is simply a measure of the magnitude of the complex field, $\phi$. This can be written straightforwardly as
%
\beq
\phi^{*}\phi = \sum_{\vec{x},\tau} a^{d} d\tau \phi^{*}_{r,\tau}\phi_{r,\tau}
\eeq
%
While this is not an incredibly useful quantity to calculate, it's very simple and can be used in some early testing to ensure stability of the method and check against other results. 

\subsubsection{Complexification}
When we apply the complexification to the real fields, $\phi_{a/b}\rightarrow \phi_{a/b}^{R} + i \phi_{a/b}^{I}$, we get:
%
\bea
\phi^{*}\phi &=&\frac{1}{2} \sum_{\vec{x},\tau} a^{d}\sum_{a=1}^{2} d\tau \left(\phi_{a,r}\phi_{a,r}\right) \nonumber \\
&=& \frac{1}{2}\sum_{\vec{x},\tau} a^{d} \sum_{a=1}^{2}  d\tau \big[ \big( (\phi_{a,r}^{R})^{2}- (\phi_{a,r}^{I})^{2} \big) + i \phi_{a,r}^{R}\phi_{a,r}^{I}\big]
\eea
%
This was done in Mathematica.

%%%%%%%%%%%%%%%%%%%%%%%%%%%%%%%%%%%%%%%%%%%%%
%%%%%%%%%%%%%%%%%%%%%%%%%%%%%%%%%%%%%%%%%%%%%
\subsection{Angular Momentum}
We can compute the angular momentum by taking a derivative of the lattice action with respect to the rotation frequency. Since the only part of the lattice action that depends on $\bar{\omega}_{z}$ is $S_{\omega}$, we can ignore the rest of the action:
%
\bea
\hat{L_{z}}[\phi_{i}] &=& - \frac{\partial}{\partial \bar{\omega}_{z}} S_{lat} = \frac{\partial}{\partial \bar{\omega}_{z}} S_{\omega}  \nonumber \\
%
&=& \frac{\partial}{\partial \bar{\omega}_{z}}  \sum_{\vec{x},\tau} a^{d} \left( i \bar{\omega}_{z} \left(\bar{x}\phi_{r}^{*}\phi_{r - \hat{\tau}} - \bar{x} \phi_{r}^{*}\phi_{r - \hat{y} - \hat{\tau}} - \bar{y} \phi_{r}^{*}\phi_{r - \hat{\tau}}+ \bar{y} \phi_{r}^{*}\phi_{r - \hat{x} - \hat{\tau}} \right)\right) \nonumber \\
%
&=& \sum_{\vec{x},\tau} a^{d} \left( i  \left(\bar{x}\phi_{r}^{*}\phi_{r - \hat{\tau}} - \bar{x} \phi_{r}^{*}\phi_{r - \hat{y} - \hat{\tau}} - \bar{y} \phi_{r}^{*}\phi_{r - \hat{\tau}}+ \bar{y} \phi_{r}^{*}\phi_{r - \hat{x} - \hat{\tau}} \right)\right) 
\eea
%
This can be written in terms of $S_{\omega}$:
%
\beq
\hat{L_{z}}[\phi_{i}] =  \frac{1}{\bar{\omega}_{z}} S_{\omega}
\eeq
%
\note{Check the sign here -- you may have lost a minus sign}

\subsubsection{Complexification}
When we apply the complexification to the real fields, $\phi_{a/b}\rightarrow \phi_{a/b}^{R} + i \phi_{a/b}^{I}$, we get:
%
\bea
\hat{L_{z}}[\phi_{i}] &=& \frac{1}{2} \sum_{a=1}^{2} \left[\sum_{b=1}^{2}\epsilon_{ab}\left((\bar{y} - \bar{x})\phi_{a,r}\phi_{b,r-\hat{\tau}}+ \bar{x}\phi_{a,r}\phi_{b,r - \hat{y}-\hat{\tau}} - \bar{y}\phi_{a,r}\phi_{b,r - \hat{x}-\hat{\tau}}\right)\right]\nonumber\\  
&& + i  \frac{1}{2}\sum_{a=1}^{2} \left[(\bar{x} - \bar{y}) \phi_{a,r}\phi_{a,r-\hat{\tau}}- \bar{x} \phi_{a,r}\phi_{a,r - \hat{y}-\hat{\tau}} + \bar{y} \phi_{a,r}\phi_{a,r - \hat{x}-\hat{\tau}}\right]\nonumber\\  
&=&\frac{1}{2} \sum_{a=1}^{2} \left[ (y-x) \phi_{a,r}^{R}\phi_{a,r-\hat{\tau}}^{I} + (y-x)\phi_{a,r}^{I} \phi_{a,r-\hat{\tau}}^{R} \right. \nonumber \\
&&\left.+ 
x \phi_{a,r}^{R}\phi_{a,r-\hat{\tau}-\hat{y}}^{I} + x \phi_{a,r}^{I}\phi_{a,r-\hat{\tau}-\hat{y}}^{R} - y \phi_{a,r}^{R}\phi_{a,r-\hat{\tau}-\hat{x}}^{I} - y \phi_{a,r}^{I}\phi_{a,r-\hat{\tau}-\hat{x}}^{R}\right. \nonumber \\
&&\left. + \epsilon_{ab}\sum_{b=1}^{2}\left( (x-y) \phi_{a,r}^{I}\phi_{b,r-\hat{\tau}-\hat{y}}^{I} + (y-x)\phi_{a,r}^{R}\phi_{b,r-\hat{\tau}-\hat{y}}^{R} \right. \right.\nonumber \\ 
&&\left.\left. + x \phi_{a,r}^{R}\phi_{b,r-\hat{\tau}-\hat{y}}^{R}-x \phi_{a,r}^{I}\phi_{b,r-\hat{\tau}-\hat{y}}^{I} + y\phi_{a,r}^{I}\phi_{b,r-\hat{\tau}-\hat{x}}^{I} - y\phi_{a,r}^{R}\phi_{b,r-\hat{\tau}-\hat{x}}^{R} \right) \right] \nonumber \\
%
&=&\frac{i}{2} \sum_{a=1}^{2} \left[ (x-y) \phi_{a,r}^{R}\phi_{a,r-\hat{\tau}}^{R} + (y-x) \phi_{a,r}^{I}\phi_{a,r-\hat{\tau}}^{I} - x \phi_{a,r}^{R}\phi_{a,r-\hat{\tau}-\hat{y}}^{R} \right.\nonumber \\ 
&&\left.+ x \phi_{a,r}^{I}\phi_{a,r-\hat{\tau}-\hat{y}}^{I} + y \phi_{a,r}^{R}\phi_{a,r-\hat{\tau}-\hat{x}}^{R} - y \phi_{a,r}^{I}\phi_{a,r-\hat{\tau}-\hat{x}}^{I}\right.\nonumber \\ 
&&\left. + \epsilon_{ab}\sum_{b=1}^{2}\left(  (y-x)\phi_{a,r}^{R}\phi_{b,r-\hat{\tau}}^{I} + (y-x)\phi_{a,r}^{I}\phi_{b,r-\hat{\tau}}^{R} + x \phi_{a,r}^{R}\phi_{b,r-\hat{\tau}-\hat{y}}^{I}  \right. \right.\nonumber \\ 
&&\left. \left.+ x \phi_{a,r}^{I}\phi_{b,r-\hat{\tau}-\hat{y}}^{R} - y \phi_{a,r}^{R}\phi_{b,r-\hat{\tau}-\hat{x}}^{I} - y \phi_{a,r}^{I}\phi_{b,r-\hat{\tau}-\hat{x}}^{R}
\right)\right]
%
\eea
%

%%%%%%%%%%%%%%%%%%%%%%%%%%%%%%%%%%%%%%%%%%%%%
%%%%%%%%%%%%%%%%%%%%%%%%%%%%%%%%%%%%%%%%%%%%%
\subsection{Harmonic Trapping Potential Energy}
As is likely obvious by now, the way to calculate the potential energy due to the harmonic trap is by taking a derivative with respect to  $\bar{\omega}_{\mathrm{tr}}$. The only part of the action that depends on $\bar{\omega}_{\mathrm{tr}}$ is $S_{\mathrm{tr}}$:
%
\bea
\hat{V}_{\mathrm{tr}}[\phi_{i}] &=&- \frac{\partial}{\partial \bar{\omega}_{\mathrm{tr}}} S_{lat}  =  \frac{\partial}{\partial \bar{\omega}_{\mathrm{tr}}}S_{\mathrm{tr}} \nonumber \\
& = &  \frac{\partial}{\partial \bar{\omega}_{\mathrm{tr}}} \sum_{\vec{x},\tau}a^{d}\left( \frac{\bar{m}}{2} \bar{\omega}_{\mathrm{tr}}^{2} \bar{r}_{\perp}^{2}\phi_{r}^{*}\phi_{r - \hat{\tau}} \right) = \sum_{\vec{x},\tau}a^{d}\left( \frac{\bar{m}}{2} (2 \bar{\omega}_{\mathrm{tr}}) \bar{r}_{\perp}^{2}\phi_{r}^{*}\phi_{r - \hat{\tau}} \right) \nonumber \\
& =& \sum_{\vec{x},\tau}a^{d}\left(\bar{m} \bar{\omega}_{\mathrm{tr}} \bar{r}_{\perp}^{2}\phi_{r}^{*}\phi_{r - \hat{\tau}} \right)
\eea
%
In terms of  $S_{\mathrm{tr}}$, this is
%
\beq
V_{\mathrm{tr}}  = \frac{2}{\bar{\omega}_{\mathrm{tr}}} S_{\mathrm{tr}}
\eeq
%
\note{Check the sign here as well}


\subsubsection{Complexification}
When we apply the complexification to the real fields, $\phi_{a/b}\rightarrow \phi_{a/b}^{R} + i \phi_{a/b}^{I}$, we get:
%
\bea
V_{\mathrm{tr}}  &=&\sum_{\vec{x},\tau}a^{d}\frac{\bar{m}}{2} \bar{\omega}_{\mathrm{tr}} \bar{r}_{\perp}^{2}\sum_{a=1}^{2}\left[ \phi_{a,r}\phi_{a,r-\hat{\tau}} + i \sum_{b=1}^{2}\epsilon_{ab}  \phi_{a,r}\phi_{b,r-\hat{\tau}}\right] \nonumber \\
%
&=&\frac{\bar{m}}{2} \bar{\omega}_{\mathrm{tr}} \bar{r}_{\perp}^{2}\sum_{\vec{x},\tau}a^{d}\sum_{a=1}^{2}\left[ 
\phi_{a,r}^{R}\phi_{a,r-\hat{\tau}}^{R} - \phi_{a,r}^{I}\phi_{a,r-\hat{\tau}}^{I} 
- \epsilon_{ab}(\phi_{a,r}^{R}\phi_{a,r-\hat{\tau}}^{I}+\phi_{a,r}^{I}\phi_{a,r-\hat{\tau}}^{R}) \right] \nonumber \\
&& + i \frac{\bar{m}}{2} \bar{\omega}_{\mathrm{tr}} \bar{r}_{\perp}^{2} \sum_{\vec{x},\tau}a^{d}\sum_{a=1}^{2}\left[ 
\phi_{a,r}^{R}\phi_{a,r-\hat{\tau}}^{I}+ \phi_{a,r}^{I}\phi_{a,r-\hat{\tau}}^{R}
+ \epsilon_{ab}(\phi_{a,r}^{R}\phi_{a,r-\hat{\tau}}^{R} - \phi_{a,r}^{I}\phi_{a,r-\hat{\tau}}^{I})
\right]
\eea
%


%%%%%%%%%%%%%%%%%%%%%%%%%%%%%%%%%%%%%%%%%%%%%
%%%%%%%%%%%%%%%%%%%%%%%%%%%%%%%%%%%%%%%%%%%%%
\subsection{Interaction Potential Energy}
The interaction also generates a potential energy, and we can calculate that by taking a derivative as well. The interaction parameter, $\bar{\lambda}$, is only present in the interaction term, $S_{\mathrm{int}}$:
%
\bea
\hat{V}_{\mathrm{int}}[\phi_{i}] &=&- \frac{\partial}{\partial \bar{\lambda}} S_{lat}  = - \frac{\partial}{\partial \bar{\lambda}}S_{\mathrm{int}} \nonumber \\
&=& -  \frac{\partial}{\partial \bar{\lambda}}\sum_{\vec{x},\tau}a^{d}\left( \bar{\lambda}\left(\phi_{r}^{*}\phi_{r - \hat{\tau}}\right)^{2}\right) = -  \sum_{\vec{x},\tau}a^{d}\left(\phi_{r}^{*}\phi_{r - \hat{\tau}}\right)^{2}. 
\eea
%
In terms of $S_{\bar{\lambda}}$, this is:
%
\beq
\hat{V}_{\mathrm{int}}= \frac{1}{\bar{\lambda}}S_{\bar{\lambda}}
\eeq
%

\subsubsection{Complexification}
When we apply the complexification to the real fields, $\phi_{a/b}\rightarrow \phi_{a/b}^{R} + i \phi_{a/b}^{I}$, we get:
%
\bea
\hat{V}_{\mathrm{int}}& =& \frac{1}{4}\sum_{\vec{x},\tau}a^{d}\sum_{a,b=1}^{2}\left[2 \phi_{a,r} \phi_{a,r-\hat{\tau}} \phi_{b,r}\phi_{b,r-\hat{\tau}} -\phi_{a,r}^{2} \phi_{b,r-\hat{\tau}}^{2} \right] \nonumber \\
&& +i \frac{ \bar{\lambda}}{2}\sum_{\vec{x},\tau}a^{d}\sum_{a,b=1}^{2}\left[\epsilon_{ab} \left( \phi_{a,r}^{2} \phi_{a,r-\hat{\tau}} \phi_{b,r-\hat{\tau}}-\phi_{a,r} \phi_{b,r}\phi_{a,r-\hat{\tau}}^{2} \right) \right] \nonumber \\
%
&=& \frac{1}{4}\sum_{\vec{x},\tau}a^{d}\sum_{a,b=1}^{2}\left[ 4 \phi_{a,r}^{R}\phi_{a,r}^{I}\phi_{b,r-\hat{\tau}}^{R}\phi_{b,r-\hat{\tau}}^{I}  \right.\nonumber \\
&& \left. + \phi_{a,r}^{R}\phi_{a,r}^{R}\phi_{b,r-\hat{\tau}}^{I}\phi_{b,r-\hat{\tau}}^{I} + \phi_{a,r}^{I}\phi_{a,r}^{I}\phi_{b,r-\hat{\tau}}^{R}\phi_{b,r-\hat{\tau}}^{R}
\right. \nonumber \\
&&\left.- \phi_{a,r}^{R}\phi_{a,r}^{R}\phi_{b,r-\hat{\tau}}^{R}\phi_{b,r-\hat{\tau}}^{R}  - \phi_{a,r}^{I}\phi_{a,r}^{I}\phi_{b,r-\hat{\tau}}^{I}\phi_{b,r-\hat{\tau}}^{I}\right. \nonumber \\
&&\left.+ 2 \phi_{a,r}^{R}\phi_{a,r-\hat{\tau}}^{R}\phi_{b,r}^{R}\phi_{b,r-\hat{\tau}}^{R} + 2 \phi_{a,r}^{I}\phi_{a,r-\hat{\tau}}^{I}\phi_{b,r}^{I}\phi_{b,r-\hat{\tau}}^{I}\right. \nonumber \\
&& \left. - 2 \phi_{a,r}^{R}\phi_{a,r-\hat{\tau}}^{R}\phi_{b,r}^{I}\phi_{b,r-\hat{\tau}}^{I}- 2 \phi_{a,r}^{I}\phi_{a,r-\hat{\tau}}^{I}\phi_{b,r}^{R}\phi_{b,r-\hat{\tau}}^{R}\right. \nonumber \\
&& \left.- 2 \phi_{a,r}^{R}\phi_{a,r-\hat{\tau}}^{I}\phi_{b,r}^{R}\phi_{b,r-\hat{\tau}}^{I}  - 2 \phi_{a,r}^{I}\phi_{a,r-\hat{\tau}}^{R}\phi_{b,r}^{I}\phi_{b,r-\hat{\tau}}^{R} \right.\nonumber \\
&& \left.- 2 \phi_{a,r}^{R}\phi_{a,r-\hat{\tau}}^{I}\phi_{b,r}^{I}\phi_{b,r-\hat{\tau}}^{R} - 2\phi_{a,r}^{I}\phi_{a,r-\hat{\tau}}^{R}\phi_{b,r}^{R}\phi_{b,r-\hat{\tau}}^{I} \right. \nonumber \\
&& \left.+ \epsilon_{ab}(\phi_{a,r}^{R}\phi_{a,r-\hat{\tau}}^{R}\phi_{a,r-\hat{\tau}}^{R} \phi_{b,r}^{I}+\phi_{a,r}^{I}\phi_{a,r-\hat{\tau}}^{R}\phi_{a,r-\hat{\tau}}^{R} \phi_{b,r}^{R} \right. \nonumber \\
&& \left. - \phi_{a,r}^{R} \phi_{a,r-\hat{\tau}}^{I}\phi_{a,r-\hat{\tau}}^{I}\phi_{b,r}^{I}- \phi_{a,r}^{I} \phi_{a,r-\hat{\tau}}^{I}\phi_{a,r-\hat{\tau}}^{I}\phi_{b,r}^{R} \right.\nonumber \\
&& \left.+ \phi_{a,r}^{I}\phi_{a,r}^{I}\phi_{a,r-\hat{\tau}}^{I}\phi_{b,r-\hat{\tau}}^{R}  + \phi_{a,r}^{I}\phi_{a,r}^{I}\phi_{a,r-\hat{\tau}}^{R}\phi_{b,r-\hat{\tau}}^{I} \right.\nonumber \\
&& \left.- \phi_{a,r}^{R}\phi_{a,r}^{R}\phi_{a,r-\hat{\tau}}^{I}\phi_{b,r-\hat{\tau}}^{R} - \phi_{a,r}^{R}\phi_{a,r}^{R}\phi_{a,r-\hat{\tau}}^{R}\phi_{b,r-\hat{\tau}}^{I}\right.\nonumber \\
&& \left. + 2 \phi_{a,r}^{R}\phi_{b,r}^{R}\phi_{a,r-\hat{\tau}}^{R}\phi_{a,r-\hat{\tau}}^{I} + 2\phi_{a,r}^{R}\phi_{a,r}^{I}\phi_{a,r-\hat{\tau}}^{I}\phi_{b,r-\hat{\tau}}^{I}\right.\nonumber \\
&& \left. - 2\phi_{a,r}^{R} \phi_{a,r}^{I}\phi_{a,r-\hat{\tau}}^{R}\phi_{b,r-\hat{\tau}}^{R}
 - 2 \phi_{a,r}^{I}\phi_{b,r}^{I} \phi_{a,r-\hat{\tau}}^{R}\phi_{a,r-\hat{\tau}}^{I})
\right] \nonumber \\
&& + i  \frac{1}{4}\sum_{\vec{x},\tau}a^{d}\sum_{a,b=1}^{2}\left[ \right.\nonumber \\
&& \left. 2 \phi_{a,r}^{I} \phi_{a,r}^{I}\phi_{b,r-\hat{\tau}}^{R} \phi_{b,r-\hat{\tau}}^{I} - 2\phi_{a,r}^{R} \phi_{a,r}^{R}\phi_{b,r-\hat{\tau}}^{R} \phi_{b,r-\hat{\tau}}^{I} \right.\nonumber \\
&& \left. + 2 \phi_{a,r}^{R} \phi_{a,r}^{I}\phi_{b,r-\hat{\tau}}^{I}\phi_{b,r-\hat{\tau}}^{I} - 2 \phi_{a,r}^{R} \phi_{a,r}^{I}\phi_{b,r-\hat{\tau}}^{R}\phi_{b,r-\hat{\tau}}^{R}\right.\nonumber \\
&& \left. + 2 \phi_{a,r}^{R}\phi_{b,r}^{R}\phi_{a,r-\hat{\tau}}^{I}\phi_{b,r-\hat{\tau}}^{R} - 2 \phi_{a,r}^{I}\phi_{b,r}^{I}\phi_{a,r-\hat{\tau}}^{I}\phi_{b,r-\hat{\tau}}^{R}\right.\nonumber \\
&&\left. + 2\phi_{a,r}^{R}\phi_{b,r}^{R} \phi_{a,r-\hat{\tau}}^{R}\phi_{b,r-\hat{\tau}}^{I} - 2\phi_{a,r}^{I} \phi_{b,r}^{I} \phi_{a,r-\hat{\tau}}^{R}\phi_{b,r-\hat{\tau}}^{I} 
\right.\nonumber \\
&& \left. +2\phi_{a,r}^{R}\phi_{b,r}^{I}  \phi_{a,r-\hat{\tau}}^{R}\phi_{b,r-\hat{\tau}}^{R} - 2\phi_{a,r}^{R}\phi_{b,r}^{I}\phi_{a,r-\hat{\tau}}^{I}\phi_{b,r-\hat{\tau}}^{I}\right.\nonumber \\
&& \left. + 2 \phi_{a,r}^{I}\phi_{b,r}^{R} \phi_{a,r-\hat{\tau}}^{R} \phi_{b,r-\hat{\tau}}^{R}- 2 \phi_{a,r}^{I}\phi_{b,r}^{R}\phi_{a,r-\hat{\tau}}^{I}\phi_{b,r-\hat{\tau}}^{I}\right.\nonumber \\
&& \left. + \epsilon_{ab}(
\phi_{a,r}^{R}\phi_{a,r}^{R}\phi_{a,r-\hat{\tau}}^{R}\phi_{b,r-\hat{\tau}}^{R} - \phi_{a,r}^{R}\phi_{a,r}^{R}\phi_{a,r-\hat{\tau}}^{I}\phi_{b,r-\hat{\tau}}^{I} \right.\nonumber \\
&& \left. +\phi_{a,r}^{I}\phi_{a,r}^{I}\phi_{a,r-\hat{\tau}}^{I} \phi_{b,r-\hat{\tau}}^{I}- \phi_{a,r}^{I}\phi_{a,r}^{I}\phi_{a,r-\hat{\tau}}^{R}\phi_{b,r-\hat{\tau}}^{R}\right.\nonumber \\
&& \left. +\phi_{a,r}^{I} \phi_{b,r}^{I}\phi_{a,r-\hat{\tau}}^{R}\phi_{a,r-\hat{\tau}}^{R} - \phi_{a,r}^{R}\phi_{b,r}^{R}\phi_{a,r-\hat{\tau}}^{R}\phi_{a,r-\hat{\tau}}^{R}\right.\nonumber \\
&& \left. + \phi_{a,r}^{R}\phi_{b,r}^{R} \phi_{a,r-\hat{\tau}}^{I}\phi_{a,r-\hat{\tau}}^{I}- \phi_{a,r}^{I}\phi_{b,r}^{I}\phi_{a,r-\hat{\tau}}^{I}\phi_{a,r-\hat{\tau}}^{I} \right.\nonumber \\
&& \left. + 2 \phi_{a,r}^{R}\phi_{b,r}^{I}\phi_{a,r-\hat{\tau}}^{R} \phi_{a,r-\hat{\tau}}^{I} + 2 \phi_{a,r}^{I}\phi_{b,r}^{R}\phi_{a,r-\hat{\tau}}^{R}\phi_{a,r-\hat{\tau}}^{I}\right.\nonumber \\
&& \left. - 2 \phi_{a,r}^{R} \phi_{a,r}^{I}\phi_{a,r-\hat{\tau}}^{R}\phi_{b,r-\hat{\tau}}^{I} - 2 \phi_{a,r}^{R} \phi_{a,r}^{I}\phi_{a,r-\hat{\tau}}^{I}\phi_{b,r-\hat{\tau}}^{R})
\right]
\eea
%

%%%%%%%%%%%%%%%%%%%%%%%%%%%%%%%%%%%%%%%%%%%%%
%%%%%%%%%%%%%%%%%%%%%%%%%%%%%%%%%%%%%%%%%%%%%
\subsection{Kinetic Energy}
Of course, in addition to the potential energies, our system has a kinetic energy, which we can calculate. The kinetic energy is actually one of the terms in the action, and we don't need to perform any derivatives to calculate it:
%
\beq
\hat{T} = S_{\del} = \sum_{\vec{x},\tau}a^{d}\left( \frac{1}{2 \bar{m}} \sum_{j=1}^{d} \left(2 \phi_{r}^{*}\phi_{r}  - \phi_{r}^{*}\phi_{r - \hat{j}} - \phi_{r}^{*}\phi_{r + \hat{j}}\right)\right)
\eeq
%

%\note{Talk through this with Lukas -- are there any subtleties here?}

\subsubsection{Complexification}
%
\bea
\hat{T} &= &\frac{1}{4\bar{m}}\sum_{\vec{x},\tau}a^{d}\sum_{j=1}^{d}\sum_{a=1}^{2}\left[ 2 \phi_{a,r}^{2} - (\phi_{a}\phi_{a,r-\hat{j}}+ \phi_{a}\phi_{a,r+\hat{j}}) \right. \nonumber \\
&& \left.- i \sum_{b=1}^{2}\epsilon_{ab}\left(\phi_{a,r}\phi_{b,r-\hat{j}} +\phi_{a,r}\phi_{b,r+\hat{j}}  \right)\right] \nonumber \\
&=& \frac{1}{4\bar{m}}\sum_{\vec{x},\tau}a^{d}\sum_{j=1}^{d}\sum_{a=1}^{2}\left[ 2\phi_{a,r}^{R}\phi_{a,r}^{R} - 2 \phi_{a,r}^{I}\phi_{a,r}^{I} \right. \nonumber \\
&& \left.+ \phi_{a,r}^{I} \phi_{a,r-\hat{j}}^{I}+ \phi_{a,r}^{I} \phi_{a,r+\hat{j}}^{I} -\phi_{a,r}^{R}\phi_{a,r-\hat{j}}^{R}-\phi_{a,r}^{R}\phi_{a,r+\hat{j}}^{R}\right. \nonumber \\
&& \left. + \epsilon_{ab} \left( \phi_{a,r}^{R}\phi_{b,r-\hat{j}}^{I} + \phi_{a,r}^{R}\phi_{b,r+\hat{j}}^{I} + \phi_{a,r}^{I}\phi_{b,r-\hat{j}}^{R} + \phi_{a,r}^{I}\phi_{b,r+\hat{j}}^{R} \right) \right]\nonumber \\
&& + \frac{i}{4\bar{m}}\sum_{\vec{x},\tau}a^{d}\sum_{j=1}^{d}\sum_{a=1}^{2}\left[ 4 \phi_{a,r}^{R}\phi_{a,r}^{I} - \phi_{a,r}^{R}\phi_{a,r-\hat{j}}^{I} - \phi_{a,r}^{R}\phi_{a,r+\hat{j}}^{I}  \right. \nonumber \\
&& \left.- \phi_{a,r}^{I}\phi_{a,r-\hat{j}}^{R}- \phi_{a,r}^{I}\phi_{a,r+\hat{j}}^{R} + \epsilon_{ab} \left(\phi_{a,r}^{I}\phi_{b,r-\hat{j}}^{I}   + \phi_{a,r}^{I}\phi_{b,r+\hat{j}}^{I}  \right. \right. \nonumber \\
&& \left. \left. - \phi_{a,r}^{R}\phi_{b,r-\hat{j}}^{R} - \phi_{a,r}^{R}\phi_{b,r+\hat{j}}^{R} \right)\right]
\eea
%

%\subsection{Total Energy}
%The total energy is often calculated using the Hamitonian. We can also calculate it by summing our energy terms. While in the action, we subtract potential energy from kinetic energy, in the Hamiltonian, we add it.
%\note{We see here that particle number plays a role in the energy of the system -- we can add or remove particles to change the energy. This is part of why it's such a pain to balance all our parameters. Is that right? Or does the particle number contribute to the energy as well? As a sort of chemical potential energy??}
\subsection{Action}
We see from the above work that the full action can actually be calculated by summing many of the other observables.
%
\bea
S_{\mathrm{lat}} &= & S_{\mu} + S_{\del} - S_{\mathrm{tr}} - S_{\omega} + S_{\mathrm{int}} \nonumber \\
&=& \hat{N} + \phi^{*}\phi + \hat{T} - \frac{\bar{\omega_{\mathrm{tr}}}}{2} \hat{V}_{\mathrm{tr}} - \bar{\omega}_{z} \hat{L}_{z} + \bar{\lambda} \hat{V}_{\mathrm{int}}
\eea
%

\subsubsection{Complexification}
As we have already developed a complexified expression for each of the observables that make up the action, we won't re-derive the expressions here. We can construct a complexified action from the observables already derived above.

\subsection{Total Energy}
Traditionally, in stat mech calculations, you can calculate the energy by taking a derivative of the partition function with respect to the inverse temperature, $\beta$. In lattice units, $\beta = N_{\tau}d\tau$, so we can rewrite our lattice action in terms of $\beta$. We cancelled out factors of $d\tau$ and $a$, so we need to put those back in:
%
\bea
S_{\text{lat}} &=& \sum_{\vec{x},\tau}a^{d} \left[ \phi_{r}^{*}\phi_{r} 
- e^{ d\tau \mu}\phi_{r}^{*}\phi_{r - \hat{\tau}} 
- \frac{d\tau}{2 m a^{2}} \sum_{j=1}^{d} \left(\phi_{r}^{*}\phi_{r - \hat{j}} - 2 \phi_{r}^{*}\phi_{r} + \phi_{r}^{*}\phi_{r + \hat{j}}\right)
- \frac{m}{2} d\tau \omega_{\mathrm{tr}}^{2}r^{2}\phi_{r}^{*}\phi_{r - \hat{\tau}}\right. \nonumber \\
&& \left.  + i d\tau \omega_{z} \left(\frac{x}{a} \phi_{r}^{*}\phi_{r - \hat{y} - \hat{\tau}} - \frac{x}{a}\phi_{r}^{*}\phi_{r - \hat{\tau}} - \frac{y}{a} \phi_{r}^{*}\phi_{r - \hat{x} - \hat{\tau}} + \frac{y}{a}\phi_{r}^{*}\phi_{r - \hat{\tau}}\right)
+ d\tau \lambda\left(\phi_{r}^{*}\phi_{r - \hat{\tau}}\right)^{2}\right]
\eea
%
Note that only one term in the action has no dependence on $d \tau$, and it came from a derivative with respect to $d\tau$
\note{Possibly, you have to walk this back one more step... to before you combined terms to get the exponential. It may not make a big difference, and right now we're not very interested in this observable, but look into it.}

If we multiply the whole thing by $1 = \frac{N_{\tau}}{N_{\tau}}$ (and use the same trick on the $d\tau \mu$ in the exponential), we can see where $\beta$ appears:
%
\bea
S_{\text{lat}} &=& \frac{1}{N_{\tau}} \sum_{\vec{x},\tau}a^{d} \left[ N_{\tau} \phi_{r}^{*}\phi_{r} 
- N_{\tau}e^{\beta \mu/N_{\tau}}\phi_{r}^{*}\phi_{r - \hat{\tau}} 
- \frac{\beta}{2 m a^{2}} \sum_{j=1}^{d} \left(\phi_{r}^{*}\phi_{r - \hat{j}} - 2 \phi_{r}^{*}\phi_{r} + \phi_{r}^{*}\phi_{r + \hat{j}}\right)
- \frac{m}{2} \beta \omega_{\mathrm{tr}}^{2}r^{2}\phi_{r}^{*}\phi_{r - \hat{\tau}}\right. \nonumber \\
&& \left.  + i \beta \omega_{z} \left(\frac{x}{a} \phi_{r}^{*}\phi_{r - \hat{y} - \hat{\tau}} - \frac{x}{a}\phi_{r}^{*}\phi_{r - \hat{\tau}} - \frac{y}{a} \phi_{r}^{*}\phi_{r - \hat{x} - \hat{\tau}} + \frac{y}{a}\phi_{r}^{*}\phi_{r - \hat{\tau}}\right)
+ \beta \lambda\left(\phi_{r}^{*}\phi_{r - \hat{\tau}}\right)^{2}\right]
\eea
%
We then take a derivative with respect to $\beta$, giving us:
%
\bea
\hat{E} &=& -\frac{1}{N_{\tau}} \sum_{\vec{x},\tau}a^{d} \left[\mu e^{\beta \mu/N_{\tau}}\phi_{r}^{*}\phi_{r - \hat{\tau}} 
+ \frac{1}{2 m a^{2}} \sum_{j=1}^{d} \left(\phi_{r}^{*}\phi_{r - \hat{j}} - 2 \phi_{r}^{*}\phi_{r} + \phi_{r}^{*}\phi_{r + \hat{j}}\right)
+ \frac{m}{2}  \omega_{\mathrm{tr}}^{2}r^{2}\phi_{r}^{*}\phi_{r - \hat{\tau}}\right. \nonumber \\
&& \left.  - i  \omega_{z} \left(\frac{x}{a} \phi_{r}^{*}\phi_{r - \hat{y} - \hat{\tau}} - \frac{x}{a}\phi_{r}^{*}\phi_{r - \hat{\tau}} - \frac{y}{a} \phi_{r}^{*}\phi_{r - \hat{x} - \hat{\tau}} + \frac{y}{a}\phi_{r}^{*}\phi_{r - \hat{\tau}}\right)
- \lambda\left(\phi_{r}^{*}\phi_{r - \hat{\tau}}\right)^{2}\right]
\eea
%
This can be written in terms of our other observables, and we bring back our parameters in lattice units:
%
\bea
\hat{E} &=& -\frac{1}{N_{\tau} d\tau} \sum_{\vec{x},\tau}a^{d} \left[\bar{\mu}\left( \phi_{r}^{*}\phi_{r} - S_{\mu}\right)
- S_{\del} + S_{\mathrm{tr}} + S_{\omega} - S_{\mathrm{int}}\right] \nonumber \\
&=& \sum_{\vec{x},\tau}a^{d} \left[ \frac{1}{N_{\tau} d\tau} \bar{\mu} \hat{N} + \frac{1}{N_{\tau} d\tau}  \hat{T}
- \frac{\bar{\omega}_{\mathrm{tr}}}{2} \hat{V}_{\mathrm{tr}} -  \bar{\omega}_{z}  \hat{L}_{z} +  \bar{\lambda} \hat{V}_{\mathrm{int}}\right] 
\eea
%

\subsubsection{Complexification}
Just as with the action, we can express the complexified total energy using 

% \note{OLD:
%The expectation value of the density is defined as
%
%\beq
%\langle E \rangle = \frac{-\partial \ln \CZ}{\partial \beta}= \frac{-\partial \ln (e^{-S})}{\partial \beta} = \frac{\partial S}{\partial \beta}.
%\eeq
%
%Since we have written the action in terms of $\beta$ as 
%
%\beq
%S \rightarrow \frac{1}{N_{\tau}}\sum_{x,\tau}\Delta x^{d}\beta \left[\left(\partial_{\tau}  - \frac{1}{2m}\del^{2} - \mu- i \omega_{z}(x \partial_{y} - y \partial_{x})  - \frac{m\omega_{\text{trap}}^{2}}{2}(x^{2}+y^{2}) \right)\phi + \lambda (\phi^{*}\phi)^{2}\right].
%\eeq 
%
%We can see that the action is linear in $\beta$. Therefore, we can simply calculate the energy in the following way:
%\beq
%\langle E \rangle = \frac{S}{ \beta}
%\eeq
% Is this wrong? The derivation I did above isn't quite $S/\beta$...}

\subsection{Circulation}
The circulation is defined as
%
\bea
\Gamma[l] &=& \frac{1}{2 \pi}\oint_{l \times l}dx \left(\theta_{\tau,x+j} - \theta_{\tau,x}\right) \\
\theta_{t,x} &=& \arctan\left(\frac{\text{Im}[\phi_{\tau,x}]}{\text{Re}[\phi_{\tau,x}]} \right).
\eea
%
Given that our field, $\phi$, is broken into 4 components ($\phi_{1/2}^{R/I}$, we need to rewrite this quantity in terms of those components:
%
\bea
\phi &=& \frac{1}{\sqrt{2}}\left(\phi_{1}^{R} + i \phi_{1}^{I} + i \phi_{2}^{R} - \phi_{2}^{I}\right) \nonumber \\
\text{Re}[\phi_{\tau,x}] & =& \frac{1}{\sqrt{2}}\left(\phi_{1,\tau,x}^{R} - \phi_{2,\tau,x}^{I}\right)\nonumber \\
\text{Im}[\phi_{\tau,x}] & = & \frac{1}{\sqrt{2}}\left(\phi_{1,\tau,x}^{I} + \phi_{2,\tau,x}^{R} \right)\nonumber \\
\theta_{\tau,x} &=& \arctan\left(\frac{\phi_{1,\tau,x}^{I} + \phi_{2,\tau,x}^{R}}{\phi_{1,\tau,x}^{R} - \phi_{2,\tau,x}^{I}} \right).
\eea
%
Note, we have the potential to encounter singularities here -- in the extremely unlikely event that $\phi_{1,\tau,x}^{R} = \phi_{2,\tau,x}^{I}$, we will have a zero in the denominator. This probably won't ever occur, but we should take it into account when troubleshooting.

We need to convert our integral to a sum, in order to calculate it on a lattice. So we choose an $l \times l$ sub-section of the lattice centered around the middle and proceed around it in a clockwise fashion.
%
\bea
\Gamma[l] &=& \frac{a}{2 \pi}\sum_{l \times l}\left(\theta_{\tau,x+j} - \theta_{\tau,x}\right) \nonumber \\
 &=&  \frac{a}{2 \pi}\sum_{l \times l}\left[\arctan \left( \frac{\phi_{1,\tau,x+\hat{j}}^{I} + \phi_{2,\tau,x+\hat{j}}^{R}}{\phi_{1,\tau,x+\hat{j}}^{R} - \phi_{2,\tau,x+\hat{j}}^{I}}\right) - \arctan\left(\frac{\phi_{1,\tau,x}^{I} + \phi_{2,t,x}^{R}}{\phi_{1,\tau,x}^{R} - \phi_{2,\tau,x}^{I}} \right) \right]
\eea
%
where $\phi_{\tau,x+j}$ is the value of the field at the next site on the loop from $\phi_{\tau,x}$ for each point along the loop.

We can then average this value over the extent of our temporal lattice to produce a value with a mean and standard error.

\end{document}