\documentclass[../../RotatingBosons.tex]{subfiles}

\begin{document}

%%%%%%%%%%%%%%%%%%%%%%%%%%%%%%%%%%%%%%%%%%%%%%%%%%%%%%%%%%
%%%%%%%%%%%%%%%%%%%% NONRELATIVISTIC ACTION %%%%%%%%%%%%%%%%%%%%%%
%%%%%%%%%%%%%%%%%%%%%%%%%%%%%%%%%%%%%%%%%%%%%%%%%%%%%%%%%%
\section{\label{NRAction} Justification for the Form of the Non-Relativistic Lattice Action}
The continuum action for bosons with a non-relativistic dispersion, a rotating external potential, a non-zero chemical potential, an external trap potential, and an interaction term is as follows:
%
\beq
S = \int_{V}\phi^{*}\left(\partial_{\tau}  - \frac{1}{2m}\del^{2} - \mu- i \omega_{z}(x \partial_{y} - y \partial_{x})  - \frac{m\omega_{\text{trap}}^{2}}{2}(x^{2}+y^{2}) \right)\phi + \lambda \int_{V}(\phi^{*}\phi)^{2}.
\eeq 
%
To convert this to a lattice action, we must first discretize the derivatives. We will use a backwards finite difference discretization for the single derivative and a central difference approximation for the double derivative, such that:
%
\bea
\partial_{i} \phi_{r} &= &\frac{1}{a}\left(\phi_{r} - \phi_{r - \hat{i}}\right)\\
\del^{2} \phi_{r} & = & \sum_{i} \frac{1}{a^{2}} \left( \phi_{r + \hat{i}} - 2 \phi_{r} + \phi_{r - \hat{i}} \right),
\eea 
%
where $r = (x,y,\tau)$ and the discretization length is $a$ for spatial derivatives and $d\tau$ for temporal ones.

In order to treat the finite chemical potential, the external trapping potential, the rotation, and the interaction we must shift our indices on the field that is acted on by these external parameters by one step in the time direction. This is to make these potentials gauge invariant in the lattice formulation. Since we have periodic boundary conditions in time, we don't have to worry about boundaries in time. 

When we go from the continuous action to the discrete action, we must account for the role of finite spacing.
%
\beq
\int \dif^{d}x \dif \tau \rightarrow \sum_{\vec{x},\tau}a^{d} d\tau.
\eeq
%
We then scale our parameters to their lattice versions, incorporating the lattice spacing, denoted by a bar:
%
\bea
\bar{x} &=& x/a \nonumber \\
\bar{y} &=& y/a \nonumber \\
\bar{r}^{2} &=& r^{2}/a^{2} \nonumber \\
\bar{\mu} &=& d\tau \mu \nonumber \\
\bar{m} &=& m a^{2}/d\tau \nonumber \\
\bar{\omega}_{\mathrm{tr}} &=& d\tau \omega_{\mathrm{tr}} \nonumber \\
\bar{\omega}_{z} &=& d\tau \omega_{z} \nonumber \\
\bar{\lambda} &=& d\tau \lambda \nonumber
\eea
%
This allows us to cancel factors of $d\tau$ and leaves us with an overall $a^{d}$ that we can divide out. Therefore, our lattice action becomes:
%
\bea
S_{\text{lat}} &=& \sum_{\vec{x},\tau}a^{d} \left[ \phi_{r}^{*}\phi_{r} -\phi_{r}^{*}\phi_{r - \hat{\tau}} - \bar{\mu} \phi_{r}^{*}\phi_{r - \hat{\tau}} - \frac{1}{2 \bar{m}} \sum_{j=1}^{d} \left(\phi_{r}^{*}\phi_{r - \hat{j}} - 2 \phi_{r}^{*}\phi_{r} + \phi_{r}^{*}\phi_{r + \hat{j}}\right)\right. \nonumber \\
&& \left. - \frac{\bar{m}}{2} \bar{\omega}_{\mathrm{tr}}^{2} \bar{r}_{\perp}^{2}\phi_{r}^{*}\phi_{r - \hat{\tau}} + i \bar{\omega}_{z} \left(\bar{x} \phi_{r}^{*}\phi_{r - \hat{y} - \hat{\tau}} - \bar{x}\phi_{r}^{*}\phi_{r - \hat{\tau}} - \bar{y} \phi_{r}^{*}\phi_{r - \hat{x} - \hat{\tau}} + \bar{y} \phi_{r}^{*}\phi_{r - \hat{\tau}}\right)\right. \nonumber \\
&& \left. +\bar{\lambda}\left(\phi_{r}^{*}\phi_{r - \hat{\tau}}\right)^{2}\right]
\eea
%
We can then combine our time derivative and our chemical potential in the following way:
%
\bea
S_{\text{lat}} &=& \sum_{\vec{x},\tau}a^{d} \left[ \phi_{r}^{*}\phi_{r} -(1+\bar{\mu})\phi_{r}^{*}\phi_{r - \hat{\tau}} - \frac{1}{2 \bar{m}} \sum_{j=1}^{d} \left(\phi_{r}^{*}\phi_{r - \hat{j}} - 2 \phi_{r}^{*}\phi_{r} + \phi_{r}^{*}\phi_{r + \hat{j}}\right)\right. \nonumber \\
&& \left. - \frac{\bar{m}}{2} \bar{\omega}_{\mathrm{tr}}^{2} \bar{r}_{\perp}^{2}\phi_{r}^{*}\phi_{r - \hat{\tau}} + i \bar{\omega}_{z} \left(\bar{x} \phi_{r}^{*}\phi_{r - \hat{y} - \hat{\tau}} - \bar{x}\phi_{r}^{*}\phi_{r - \hat{\tau}} - \bar{y} \phi_{r}^{*}\phi_{r - \hat{x} - \hat{\tau}} + \bar{y} \phi_{r}^{*}\phi_{r - \hat{\tau}}\right)\right. \nonumber \\
&& \left. +\bar{\lambda}\left(\phi_{r}^{*}\phi_{r - \hat{\tau}}\right)^{2}\right]
\eea
%
Note that to second order in the lattice size, this is equivalent to:
%
\bea
S_{\text{lat}} &=& \sum_{\vec{x},\tau}a^{d} \left[ \phi_{r}^{*}\phi_{r} -e^{\bar{\mu}}\phi_{r}^{*}\phi_{r - \hat{\tau}} - \frac{1}{2 \bar{m}} \sum_{j=1}^{d} \left(\phi_{r}^{*}\phi_{r - \hat{j}} - 2 \phi_{r}^{*}\phi_{r} + \phi_{r}^{*}\phi_{r + \hat{j}}\right)- \frac{\bar{m}}{2} \bar{\omega}_{\mathrm{tr}}^{2} \bar{r}_{\perp}^{2}\phi_{r}^{*}\phi_{r - \hat{\tau}}\right. \nonumber \\
&& \left.  + i \bar{\omega}_{z} \left(\bar{x} \phi_{r}^{*}\phi_{r - \hat{y} - \hat{\tau}} - \bar{x}\phi_{r}^{*}\phi_{r - \hat{\tau}} - \bar{y} \phi_{r}^{*}\phi_{r - \hat{x} - \hat{\tau}} + \bar{y} \phi_{r}^{*}\phi_{r - \hat{\tau}}\right)+\bar{\lambda}\left(\phi_{r}^{*}\phi_{r - \hat{\tau}}\right)^{2}\right]
\eea
%
This will be our lattice action, which we will complexify and use to evolve our system in Langevin time. To simplify, let's divide the lattice action into smaller components:
%
\beq
S_{\text{lat}} = \left(S_{\mu} + S_{\del} - S_{\text{trap}} - S_{\omega} + S_{\text{int}}\right)
\eeq
%
with
%
\bea
S_{\mu,r} & =& \sum_{\vec{x},\tau}a^{d}\left(\phi_{r}^{*}\phi_{r} -e^{\bar{\mu}}\phi_{r}^{*}\phi_{r - \hat{\tau}}\right)\\
S_{\del,r}& =&\sum_{\vec{x},\tau}a^{d}\left( \frac{1}{2 \bar{m}} \sum_{j=1}^{d} \left(2 \phi_{r}^{*}\phi_{r}  - \phi_{r}^{*}\phi_{r - \hat{j}} - \phi_{r}^{*}\phi_{r + \hat{j}}\right)\right)\\
S_{\text{trap},r} & =&\sum_{\vec{x},\tau}a^{d}\left( \frac{\bar{m}}{2} \bar{\omega}_{\mathrm{tr}}^{2} \bar{r}_{\perp}^{2}\phi_{r}^{*}\phi_{r - \hat{\tau}} \right)\\
S_{\omega,r} &  = &  \sum_{\vec{x},\tau}a^{d}\left( i \bar{\omega}_{z} \left(\bar{x}\phi_{r}^{*}\phi_{r - \hat{\tau}} - \bar{x} \phi_{r}^{*}\phi_{r - \hat{y} - \hat{\tau}} - \bar{y} \phi_{r}^{*}\phi_{r - \hat{\tau}}+ \bar{y} \phi_{r}^{*}\phi_{r - \hat{x} - \hat{\tau}} \right)\right)\\
S_{\text{int},r}& =& \sum_{\vec{x},\tau}a^{d}\left( \bar{\lambda}\left(\phi_{r}^{*}\phi_{r - \hat{\tau}}\right)^{2}\right).
\eea 
%
Note that we are restricting ourselves to two spatial dimensions at this point in the work. The extension of this method to three-dimensional rotating systems is saved for future work.

%%%%%%%%%%%%%%%%%%%%%%%%%%%%%%%%%%%%%%%%%%%
%%%%%%%%%%%%%%%COMPLEXIFICATION%%%%%%%%%%%%%%%%%
%%%%%%%%%%%%%%%%%%%%%%%%%%%%%%%%%%%%%%%%%%%

\section{\label{FirstComplexification} Writing the Complex Action in Terms of Real Fields}
This action must first be rewritten with the complex fields expressed in terms of two real fields, $\phi = \frac{1}{\sqrt{2}}\left(\phi_{1} + i \phi_{2}\right)$ and $\phi^{*} = \frac{1}{\sqrt{2}}\left(\phi_{1} - i \phi_{2}\right)$. Each piece of the action is computed below. Note from this point on, we drop the external sum over the entire spacetime lattice and its weighting ($a^{d}$), as it is the same for all components of the action.

First, the time derivative and chemical potential part of the action at any given site $\vec{x},\tau$ on the lattice:
%
\bea
S_{\mu,r} & =&\phi_{r}^{*}\phi_{r} - e^{ \bar{\mu}} \phi_{r}^{*}\phi_{r - \hat{\tau}} \nonumber\\
& =& \frac{1}{2}\left[\phi_{1,r}^{2} + \phi_{2,r}^{2} - e^{\bar{\mu}} \left(\phi_{1,r}\phi_{1,r - \hat{\tau}} + i \phi_{1,r}\phi_{2,r - \hat{\tau}} - i \phi_{2,r}\phi_{1,r - \hat{\tau}} +\phi_{2,r} \phi_{2,r - \hat{\tau}}\right) \right]\nonumber \\
& =& \frac{1}{2}\sum_{a=1}^{2}\left[\phi_{a,r}^{2} - e^{ \bar{\mu}}\phi_{a,r}\phi_{a,r - \hat{\tau}} - i e^{\bar{\mu}} \sum_{b=1}^{2}\epsilon_{ab}  \phi_{a,r}\phi_{b,r - \hat{\tau}} \right] ,
\eea
%
where $\epsilon_{12}  = 1,\ \epsilon_{21} = -1, \epsilon_{11}=\epsilon_{22} =0 $. Next, the spatial derivative part (corresponding to the kinetic energy):
%
\bea
S_{\del,r}& =& \frac{1}{2 \bar{m}} \sum_{j=1}^{d} \left(2 \phi_{r}^{*}\phi_{r}  - \phi_{r}^{*}\phi_{r - \hat{j}} - \phi_{r}^{*}\phi_{r + \hat{j}}\right)\nonumber\\
& =& \frac{1}{4\bar{m}}\sum_{j=1}^{d} \left[2 \left(\phi_{1,r}^{2} + \phi_{2,r}^{2}\right) - \left(\phi_{1,r}\phi_{1,r + \hat{i}}  + i \phi_{1,r}\phi_{2,r + \hat{i}} - i \phi_{2,r} \phi_{1,r + \hat{i}} + \phi_{2,r}\phi_{2,r + \hat{i}}  \right) \right. \nonumber \\
&& \left. - \left(\phi_{1,r}\phi_{1,r - \hat{i}}  + i \phi_{1,r}\phi_{2,r - \hat{i}} - i \phi_{2,r} \phi_{1,r - \hat{i}} + \phi_{2,r}\phi_{2,r - \hat{i}}  \right)   \right] \nonumber\\
& =&\frac{1}{4\bar{m}}\sum_{j=1}^{d}\sum_{a=1}^{2}\left[ 2 \phi_{a,r}^{2} - (\phi_{a}\phi_{a,r-\hat{j}}+ \phi_{a}\phi_{a,r+\hat{j}}) - i \sum_{b=1}^{2}\epsilon_{ab}\left(\phi_{a,r}\phi_{b,r-\hat{j}} +\phi_{a,r}\phi_{b,r+\hat{j}}  \right)\right].
\eea
%
Then, for the part of the action due to the external trapping potential:
%
\bea
S_{\text{trap},r} & =&\frac{\bar{m}}{2} \bar{\omega}_{\mathrm{tr}}^{2} \bar{r}_{\perp}^{2}\phi_{r}^{*}\phi_{r - \hat{\tau}} \nonumber \\
&=&\frac{\bar{m}}{4} \bar{\omega}_{\mathrm{tr}}^{2} \bar{r}_{\perp}^{2}\left[\phi_{1,r} \phi_{1,r-\tau} + i \phi_{1,r} \phi_{2,r-\tau} -i \phi_{2,r} \phi_{1,r-\tau} + \phi_{2,r} \phi_{2,r-\tau}\right]\nonumber\\
& =&\frac{\bar{m}}{4} \bar{\omega}_{\mathrm{tr}}^{2} \bar{r}_{\perp}^{2}\sum_{a=1}^{2}\left[ \phi_{a,r}\phi_{a,r-\hat{\tau}} + i \sum_{b=1}^{2}\epsilon_{ab}  \phi_{a,r}\phi_{b,r-\hat{\tau}}\right]. 
\eea
%
Next, the rotational piece:
%
\bea
S_{\omega,r} &  = &    i \bar{\omega}_{z} \left(\bar{x}\phi_{r}^{*}\phi_{r - \hat{\tau}} - \bar{x} \phi_{r}^{*}\phi_{r - \hat{y} - \hat{\tau}} - \bar{y} \phi_{r}^{*}\phi_{r - \hat{\tau}}+ \bar{y} \phi_{r}^{*}\phi_{r - \hat{x} - \hat{\tau}} \right)\nonumber \\
%
& = &   \frac{\bar{\omega}_{z}}{2} \left[ (\bar{y} - \bar{x})\left(\phi_{1,r}\phi_{2,r-\hat{\tau}} -\phi_{2,r}\phi_{1,r-\hat{\tau}} \right) 
+ \bar{x}(\phi_{1,r}\phi_{2,r - \hat{y}-\hat{\tau}} - \phi_{2,r}\phi_{1,r - \hat{y}-\hat{\tau}})\right.\nonumber\\
&& \left. - \bar{y}(\phi_{1,r}\phi_{2,r - \hat{x}-\hat{\tau}} - \phi_{2,r}\phi_{1,r - \hat{x}-\hat{\tau}})\right] \nonumber\\
&& + i  \frac{\bar{\omega}_{z}}{2}\left[ ( \bar{x} - \bar{y}) (\phi_{1,r}\phi_{1,r-\hat{\tau}}+\phi_{2,r}\phi_{2,r-\hat{\tau}})
 - \bar{x} (\phi_{1,r}\phi_{1,r - \hat{y}-\hat{\tau}} + \phi_{2,r}\phi_{2,r - \hat{y}-\hat{\tau}} ) \right. \nonumber \\
 && \left. + \bar{y} (\phi_{1,r}\phi_{1,r - \hat{x}-\hat{\tau}} + \phi_{2,r}\phi_{2,r - \hat{x}-\hat{\tau}} ) \right] \nonumber\\
 &  = &    \frac{  \bar{\omega}_{z}}{2} \sum_{a=1}^{2} \left[\sum_{b=1}^{2}\epsilon_{ab}\left((\bar{y} - \bar{x})\phi_{a,r}\phi_{b,r-\hat{\tau}}+ \bar{x}\phi_{a,r}\phi_{b,r - \hat{y}-\hat{\tau}} - \bar{y}\phi_{a,r}\phi_{b,r - \hat{x}-\hat{\tau}}\right)\right]\nonumber\\  
&& + i  \frac{\bar{\omega}_{z}}{2}\sum_{a=1}^{2} \left[(\bar{x} - \bar{y}) \phi_{a,r}\phi_{a,r-\hat{\tau}}- \bar{x} \phi_{a,r}\phi_{a,r - \hat{y}-\hat{\tau}} + \bar{y} \phi_{a,r}\phi_{a,r - \hat{x}-\hat{\tau}}\right]
\eea
%
And finally, the interaction term in the action:
%
\bea
S_{\text{int},r}& =& \bar{\lambda}(\phi_{r}^{*}\phi_{r-\hat{\tau}})^{2}\nonumber \\
& =&  \frac{ \bar{\lambda}}{4}\left[ \phi_{1,r}^{2}\phi_{1,r-\hat{\tau}}^{2} +  \phi_{2,r}^{2}\phi_{2,r-\hat{\tau}}^{2}+4 \phi_{1,r}\phi_{2,r}\phi_{1,r-\hat{\tau}}\phi_{2,r-\hat{\tau}} 
- \phi_{1,r}^{2}\phi_{2,r-\hat{\tau}}^{2}-  \phi_{2,r}^{2}\phi_{1,r-\hat{\tau}}^{2}
\right]\nonumber \\
&&+ i \frac{ \bar{\lambda}}{4} \left[2 \phi_{1,r}^{2}\phi_{1,r-\hat{\tau}}\phi_{2,r-\hat{\tau}}  - 2 \phi_{2,r}^{2}\phi_{1,r-\hat{\tau}}\phi_{2,r-\hat{\tau}} 
- 2 \phi_{1,r}\phi_{2,r}\phi_{1,r-\hat{\tau}}^{2} + 2 \phi_{1,r}\phi_{2,r}\phi_{2,r-\hat{\tau}}^{2}
\right]\nonumber \\
& =& \frac{ \bar{\lambda}}{4}\sum_{a=1}^{2}\sum_{b=1}^{2}\left[2 \phi_{a,r} \phi_{a,r-\hat{\tau}} \phi_{b,r}\phi_{b,r-\hat{\tau}} -\phi_{a,r}^{2} \phi_{b,r-\hat{\tau}}^{2} \right] \nonumber \\
&& +i \frac{ \bar{\lambda}}{2}\sum_{a=1}^{2}\sum_{b=1}^{2}\left[\epsilon_{ab} \left( \phi_{a,r}^{2} \phi_{a,r-\hat{\tau}} \phi_{b,r-\hat{\tau}}-\phi_{a,r} \phi_{b,r}\phi_{a,r-\hat{\tau}}^{2} \right) \right] \nonumber.
\eea
%
We will work with the lattice action in this form in order to derive the Langevin drift function.


\end{document}