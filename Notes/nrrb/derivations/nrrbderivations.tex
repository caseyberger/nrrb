\documentclass[../../RotatingBosons.tex]{subfiles}

\begin{document}

%%%%%%%%%%%%%%%%%%%%%%%%%%%%%%%%%%%%%%%%%%%%%%%%%%%%%%%%%%
%%%%%%%%%%%%%%%%%%%% NONRELATIVISTIC ACTION %%%%%%%%%%%%%%%%%%%%%%
%%%%%%%%%%%%%%%%%%%%%%%%%%%%%%%%%%%%%%%%%%%%%%%%%%%%%%%%%%
\section{\label{NRAction} Justification for the Form of the Non-Relativistic Lattice Action}
The continuum action for bosons with a non-relativistic dispersion, a rotating external potential, a non-zero chemical potential, an external trap potential, and an interaction term is as follows:
%
\beq
S = \int_{V}\phi^{*}\left(\partial_{\tau}  - \frac{1}{2m}\del^{2} - \mu- i \omega_{z}(x \partial_{y} - y \partial_{x})  - \frac{m\omega_{\text{trap}}^{2}}{2}(x^{2}+y^{2}) \right)\phi + \lambda \int_{V}(\phi^{*}\phi)^{2}.
\eeq 
%
To convert this to a lattice action, we must first discretize the derivatives. We will use a backwards finite difference discretization for the single derivative and a central difference approximation for the double derivative, such that:
%
\bea
\partial_{i} \phi_{r} &= &\frac{1}{a}\left(\phi_{r} - \phi_{r - \hat{i}}\right)\\
\del^{2} \phi_{r} & = & \sum_{i} \frac{1}{a^{2}} \left( \phi_{r + \hat{i}} - 2 \phi_{r} + \phi_{r - \hat{i}} \right),
\eea 
%
where $r = (x,y,\tau)$ and the discretization length $a$ (lattice spacing) is $1$ for spatial derivatives and $d\tau$ for temporal ones.

In order to treat the finite chemical potential, the external trapping potential, the rotation, and the interaction we must shift our indices on the field that is acted on by $\mu$ and $\omega_{\text{trap}}$ by one step in the time direction. This is to make these potentials gauge invariant in the lattice formulation. Since we have periodic boundary conditions in time, we don't have to worry about boundaries is. Therefore, our lattice action becomes, at each lattice site, $r$:
%
\bea
S_{\text{lat},r} &=& \phi_{r}^{*}\left[\phi_{r} - \phi_{r - \hat{\tau}} - d\tau \mu \phi_{r- \hat{\tau}} - \frac{d\tau}{2m}\sum_{i = 1}^{d}\left(  \phi_{r + \hat{i}} - 2 \phi_{r} + \phi_{r - \hat{i}} \right)- \frac{d\tau m\omega_{\text{trap}}^{2}}{2}(x^{2}+y^{2}) \phi_{r-\hat{\tau}}  \right] \\
&& -\phi_{r}^{*} \left[i d\tau\omega_{z}\left(x \phi_{r-\hat{\tau}} - x\phi_{r-\hat{y}-\hat{\tau}} - y \phi_{r-\hat{\tau}} + y \phi_{r-\hat{x}-\hat{\tau}}\right)\right]+ d\tau\lambda(\phi_{r}^{*}\phi_{r-\hat{\tau}})^{2}. \nonumber
\eea
%
We can then combine our time derivative and our chemical potential in the following way:
%
\bea
S_{\text{lat},r} &=& \phi_{r}^{*}\left[\phi_{r} - (1+  d\tau \mu)\phi_{r - \hat{\tau}} - \frac{ d\tau }{2m}\sum_{i = 1}^{d}\left(  \phi_{r + \hat{i}} - 2 \phi_{r} + \phi_{r - \hat{i}}\right)  -  d\tau \frac{m\omega_{\text{trap}}^{2}}{2}(x^{2}+y^{2}) \phi_{r-\hat{\tau}}\right] \\
& &  -\phi_{r}^{*} \left[i d\tau \omega_{z}\left(x \phi_{r-\hat{\tau}} - x\phi_{r-\hat{y}-\hat{\tau}} - y \phi_{r-\hat{\tau}} + y \phi_{r-\hat{x}-\hat{\tau}}\right)\right]+d\tau \lambda(\phi_{r}^{*}\phi_{r-\hat{\tau}})^{2}. \nonumber
\eea 
%
Note that to second order in the lattice size, this is equivalent to:
%
\bea
S_{\text{lat},r} &=& \phi_{r}^{*}\left[\phi_{r} - e^{ d\tau \mu} \phi_{r - \hat{\tau}} - \frac{ d\tau }{2m}\sum_{i = 1}^{d}\left(  \phi_{r + \hat{i}} - 2 \phi_{r} + \phi_{r - \hat{i}}\right) -  d\tau \frac{m\omega_{\text{trap}}^{2}}{2}(x^{2}+y^{2}) \phi_{r-\hat{\tau}} \right]\\
& & -  i  d\tau \omega_{z} \phi_{r}^{*} \left[(x-y) \phi_{r-\hat{\tau}} - x\phi_{r-\hat{y}-\hat{\tau}} + y \phi_{r-\hat{x}-\hat{\tau}}\right]+  d\tau \lambda(\phi_{r}^{*}\phi_{r-\hat{\tau}})^{2}.\nonumber
\eea 
%
This will be our lattice action, which we will complexify and use to evolve our system in Langevin time. To simplify, let's divide the lattice action into smaller components:
%
\beq
S_{\text{lat}} = \sum_{r}\left(S_{\tau,r} + S_{\del,r} - S_{\text{trap},r} - S_{\omega,r} + S_{\text{int},r}\right)
\eeq
%
with
%
\bea
S_{\tau,r} & =&\phi_{r}^{*}\phi_{r} - e^{ d\tau \mu} \phi_{r}^{*}\phi_{r - \hat{\tau}}\\
S_{\del,r}& =& d\tau\frac{d   }{m}\phi_{r}^{*}\phi_{r} - \frac{ d\tau }{2m}\sum_{i = 1}^{d}\left( \phi_{r}^{*} \phi_{r + \hat{i}}  + \phi_{r}^{*}\phi_{r - \hat{i}}\right) \\
S_{\text{trap},r} & =& d\tau \frac{m\omega_{\text{trap}}^{2}}{2}(x^{2}+y^{2}) \phi_{r}^{*}\phi_{r-\hat{\tau}} \\
S_{\omega,r} &  = &   i  d\tau \omega_{z} \left[(x-y) \phi_{r}^{*}\phi_{r-\hat{\tau}} - x \phi_{r}^{*}\phi_{r-\hat{y}-\hat{\tau}} + y\phi_{r}^{*} \phi_{r-\hat{x}-\hat{\tau}}\right]\\
S_{\text{int},r}& =&  d\tau  \lambda(\phi_{r}^{*}\phi_{r-\hat{\tau}})^{2}.
\eea 
%
Note that we are restricting ourselves to two spatial dimensions at this point in the work. The extension of this method to three-dimensional rotating systems is saved for future work.

This action must first be rewritten with the complex fields expressed in terms of two real fields, $\phi = \frac{1}{\sqrt{2}}\left(\phi_{1} + i \phi_{2}\right)$ and $\phi^{*} = \frac{1}{\sqrt{2}}\left(\phi_{1} - i \phi_{2}\right)$. Each piece of the action is computed below:

First, the time derivative and chemical potential part of the action:
%
\bea
S_{\tau,r} & =&\phi_{r}^{*}\phi_{r} - e^{ d\tau \mu} \phi_{r}^{*}\phi_{r - \hat{\tau}} \nonumber\\
& =& \frac{1}{2}\left[\phi_{1,r}^{2} + \phi_{2,r}^{2} - e^{ d\tau \mu} \left(\phi_{1,r}\phi_{1,r - \hat{\tau}} + i \phi_{1,r}\phi_{2,r - \hat{\tau}} - i \phi_{2,r}\phi_{1,r - \hat{\tau}} +\phi_{2,r} \phi_{2,r - \hat{\tau}}\right) \right] \\
& =& \frac{1}{2}\sum_{a=1}^{2}\left[\phi_{a,r}^{2} - e^{ d\tau \mu}\phi_{a,r}\phi_{a,r - \hat{\tau}} - i e^{ d\tau \mu} \sum_{b=1}^{2}\epsilon_{ab}  \phi_{a,r}\phi_{b,r - \hat{\tau}} \right] .\nonumber
\eea
%
Next, the spatial derivative part (corresponding to the kinetic energy):
%
\bea
S_{\del,r}& =& \frac{d\tau}{2m}\left[2 d\phi_{r}^{*}\phi_{r} - \sum_{i = 1}^{d}\left( \phi_{r}^{*} \phi_{r + \hat{i}}  + \phi_{r}^{*}\phi_{r - \hat{i}}\right)\right] \nonumber\\
& =& \frac{d\tau}{4m}\left[ 2d\left(\phi_{1,r}^{2} + \phi_{2,r}^{2}\right) - \sum_{i = \pm1}^{d}\left(\phi_{1,r}\phi_{1,r + \hat{i}}  + i \phi_{1,r}\phi_{2,r + \hat{i}} - i \phi_{2,r} \phi_{1,r + \hat{i}} + \phi_{2,r}\phi_{2,r + \hat{i}}  \right) \right]\\
& =& \frac{d\tau}{4m}\sum_{a=1}^{2}\left[ 2d\phi_{a,r}^{2} - \left(\sum_{i = \pm1}^{d}\phi_{a,r}\phi_{a,r + \hat{i}} +i\sum_{b=1}^{2}\sum_{i = \pm1}^{d}\epsilon_{ab} \phi_{a,r}\phi_{b,r + \hat{i}}\right) \right].\nonumber
\eea
%
Then, for the part of the action due to the external trapping potential:
%
\bea
S_{\text{trap},r} & =&d\tau\frac{m\omega_{\text{trap}}^{2}}{2}(x^{2}+y^{2}) \phi_{r}^{*}\phi_{r-\hat{\tau}} \nonumber \\
&=& \frac{d\tau m \omega_{\text{tr}}^{2}(x^{2}+y^{2})}{4}\left[\phi_{1,r} \phi_{1,r-\tau} + i \phi_{1,r} \phi_{2,r-\tau} -i \phi_{2,r} \phi_{1,r-\tau} + \phi_{2,r} \phi_{2,r-\tau}\right]\\
& =&\frac{d\tau\omega_{\text{tr}}(x^{2}+y^{2})}{4}\sum_{a=1}^{2}\left( \phi_{a,r}\phi_{a,r-\hat{\tau}} + i \sum_{b=1}^{2}\epsilon_{ab}  \phi_{a,r}\phi_{b,r-\hat{\tau}}\right). \nonumber
\eea
%
Next, the rotational piece:
%
\bea
S_{\omega,r} &  = &   i d\tau\omega_{z} \left[(x-y) \phi_{r}^{*}\phi_{r-\hat{\tau}} - x \phi_{r}^{*}\phi_{r-\hat{y}-\hat{\tau}} + y\phi_{r}^{*} \phi_{r-\hat{x}-\hat{\tau}}\right]\nonumber \\
%
&  = &   \frac{i d\tau\omega_{z}}{2} \left[(x-y) \left(\phi_{1,r}\phi_{1,r-\hat{\tau}} + \phi_{2,r}\phi_{2,r-\hat{\tau}} \right) +i(x-y) \left(\phi_{1,r}\phi_{2,r-\hat{\tau}} + \phi_{2,r}\phi_{1,r-\hat{\tau}}  \right)\right]\nonumber \\
&&  -   \frac{i d\tau\omega_{z}}{2} \left[x \left(\phi_{1,r}\phi_{1,r - \hat{y}-\hat{\tau}} + i \phi_{1,r}\phi_{2,r - \hat{y}-\hat{\tau}} - i \phi_{2,r}\phi_{1,r - \hat{y}-\hat{\tau}} +\phi_{2,r} \phi_{2,r - \hat{y}-\hat{\tau}}\right) \right] \nonumber \\
& & + \frac{i d\tau\omega_{z}}{2} \left[y\left(\phi_{1,r}\phi_{1,r - \hat{x}-\hat{\tau}} + i \phi_{1,r}\phi_{2,r - \hat{x}-\hat{\tau}} - i \phi_{2,r}\phi_{1,r - \hat{x}-\hat{\tau}} +\phi_{2,r} \phi_{2,r - \hat{x}-\hat{\tau}}\right) \right] \\
%
&  = &   \frac{d\tau\omega_{z}}{2}\sum_{a=1}^{2} \left[\sum_{b=1}^{2}\epsilon_{ab}\left((x-y)\phi_{a,r}\phi_{b,r-\hat{\tau}}- x \phi_{a,r}\phi_{b,r - \hat{y}-\hat{\tau}}  +y \phi_{a,r}\phi_{b,r - \hat{x}-\hat{\tau}}\right)\right]\nonumber \\
&   &   +i \frac{d\tau\omega_{z}}{2}\sum_{a=1}^{2} \left[\sum_{b=1}^{2}\left((y-x)\phi_{a,r}\phi_{a,r-\hat{\tau}}+ x \phi_{a,r}\phi_{a,r - \hat{y}-\hat{\tau}}  -y \phi_{a,r}\phi_{a,r - \hat{x}-\hat{\tau}}\right)\right].\nonumber 
\eea
%
And finally, the interaction term in the action:
%
\bea
S_{\text{int},r}& =& d\tau \lambda(\phi_{r}^{*}\phi_{r-\hat{\tau}})^{2}\nonumber \\
& =& \dots\\
& =& \frac{ d\tau\lambda}{4}\sum_{a=1}^{2}\sum_{b=1}^{2}\left[2 \phi_{a,r} \phi_{a,r-\hat{\tau}} \phi_{b,r}\phi_{b,r-\hat{\tau}} -\phi_{a,r}^{2} \phi_{b,r-\hat{\tau}}^{2} \right] \nonumber \\
&& +i \frac{ d\tau\lambda}{2}\sum_{a=1}^{2}\sum_{b=1}^{2}\left[\epsilon_{ab} \left( \phi_{a,r}^{2} \phi_{a,r-\hat{\tau}} \phi_{b,r-\hat{\tau}}-\phi_{a,r} \phi_{a,r-\hat{\tau}}^{2} \phi_{b,r} \right) \right] \nonumber.
\eea
%
We will work with the lattice action in this form in order to derive the Langevin drift function.

%%%%%%%%%%%%%%%%%%%%%%%%%%%%%%%%%%%%%%%%%%%
%%%%%%%%%%%%%%%COMPLEXIFICATION%%%%%%%%%%%%%%%%%
%%%%%%%%%%%%%%%%%%%%%%%%%%%%%%%%%%%%%%%%%%%

\section{\label{FirstComplexification} Writing the Complex Action in Terms of Real Fields}
Because we want to test our free, nonrotating, and non interacting system in multiple dimensions and leave open the possibility of future work in three dimensions, we leave the dimension number $d$ general in the following derivations.

First, we take our complex field, $\phi$, and represent it as the complex sum of two real fields: $\phi = \frac{1}{\sqrt{2}}(\phi_{1} + i \phi_{2})$. For each of the action contributions, this gives us:

\bea
\text{Re}[S_{\tau, r}] & \rightarrow & \frac{1}{2}\sum_{a=1}^{2}\left(\phi_{a,r}^{2}-e^{d\tau\mu}\ \phi_{a,r}\phi_{a,r-\hat{\tau}}\right) \\
\text{Im}[S_{\tau,r}] & \rightarrow & \frac{-e^{d\tau\mu}}{2}\sum_{a,b=1}^{2}\epsilon_{ab}\phi_{a,r}\phi_{b,r-\hat{\tau}}
\eea
%
\bea
\text{Re}[S_{\del, r}] & \rightarrow &\sum_{a = 1}^{2} \left(\frac{d}{m} \phi_{a,r}^{2} - \frac{1}{4m}\sum_{i = \pm 1}^{d}\phi_{a,r}\phi_{a,r+\hat{i}}  \right)\\
\text{Im}[S_{\del, r} ]& \rightarrow &\frac{-1}{4m} \sum_{a,b = 1}^{2} \sum_{i =\pm 1}^{d}\epsilon_{ab}\phi_{a,r}\phi_{b,r+\hat{i}}
\eea
%
\bea
\text{Re}[S_{\text{trap}, r}] & \rightarrow & \frac{m}{4}\omega_{\text{trap}}^{2}\left(x^{2} + y^{2}\right)\sum_{a=1}^{2}\phi_{a,r}\phi_{a,r-\hat{\tau}}\\
\text{Im}[S_{\text{trap}, r} ]& \rightarrow &  \frac{m}{4}\omega_{\text{trap}}^{2}\left(x^{2} + y^{2}\right)\sum_{a,b=1}^{2}\epsilon_{ab}\phi_{a,r}\phi_{b,r-\hat{\tau}}
\eea
%
\bea
\text{Re}[S_{\omega, r}] & \rightarrow & \frac{\omega_{z}}{2} \sum_{a,b = 1,2}\epsilon_{ab} \left(
(y-x)\phi_{a,r} \phi_{b,r-\hat{\tau}}+ \widetilde{x}\phi_{a,r}\phi_{b,r-\hat{y}-\hat{\tau}} - \widetilde{y} \phi_{a,r}\phi_{b,r-\hat{x}-\hat{\tau}} \right)\\
\text{Im}[S_{\omega, r} ]& \rightarrow &  \frac{\omega_{z}}{2}\sum_{a = 1}^{2}\left( (x-y) \phi_{a,r}\phi_{a,r-\hat{\tau}}- \widetilde{x} \phi_{a,r}\phi_{a,r-\hat{y}-\hat{\tau}} + \widetilde{y} \phi_{a,r}\phi_{a,r-\hat{x}-\hat{\tau}}  \right) 
\eea
%
where $\widetilde{x}$ and $\widetilde{y}$ are our $x$ and $x$ coordinates shifted by the center of the trap:
%
\bea
\widetilde{x} &=& x-\frac{N_{x}-1}{2}\nonumber \\
\widetilde{y} &=& y-\frac{N_{x}-1}{2}\nonumber 
\eea
%
\bea
\text{Re}[S_{\text{int}, r}]  & \rightarrow &  \frac{\lambda}{4}  \sum_{a,b=1}^{2} \left( 2 \phi_{a,r} \phi_{b,r}\phi_{a,r-\hat{\tau}}\phi_{b,r-\hat{\tau}} - \phi_{a,r}^{2} \phi_{b,r-\hat{\tau}}^{2}\right)\\
\text{Im}[S_{\text{int}, r} ] & \rightarrow &  \frac{\lambda}{2}  \sum_{a,b=1}^{2} \epsilon_{ab}\left(\phi_{a,r}^{2}\phi_{a,r-\hat{\tau}}\phi_{b,r-\hat{\tau}} - \phi_{a,r}\phi_{b,r}\phi_{a,r-\hat{\tau}}^{2} \right),
%
\eea 
%
where $S_{j,r} = \text{Re}[S_{j,r}]+ i \text{Im}[S_{j,r}]$, and $\epsilon_{12} =1$, $\epsilon_{21} =-1$, and  $\epsilon_{11} = \epsilon_{22} = 0$.

From here, we can compute the drift function.



\section{\label{SecondComplexification} Complexifying the Real Fields}

Now we take our real fields, $\phi_{a}$, where $a = 1,2$, and rewrite them as two complex fields: $\phi_{a} = \phi_{a}^{R} + i \phi_{a}^{I}$

The time derivative piece, $S_{\tau,r}$ becomes:
\bea
S_{\tau, r}^{R} & \rightarrow & \frac{1}{2}\sum_{a=1}^{2}\left( (\phi_{a,r}^{R})^{2}-(\phi_{a,r}^{I})^{2} -e^{d\tau\mu} \phi_{a,r}^{R}\phi_{a,r-\hat{\tau}}^{R} + e^{d\tau\mu}\phi_{a,r}^{I}\phi_{a,r-\hat{\tau}}^{I}\right) \\
& & + \frac{1}{2}\sum_{a,b=1}^{2}\epsilon_{ab}\left(e^{d\tau\mu}\phi_{a,r}^{R}\phi_{b,r-\hat{\tau}}^{I}+e^{d\tau\mu}\phi_{a,r}^{I}\phi_{b,r-\hat{\tau}}^{R} \right) \nonumber \\
S_{\tau,r}^{I} & \rightarrow & \frac{1}{2}\sum_{a=1}^{2} \left( 2 \phi_{a,r}^{R}\phi_{a,r}^{I} -e^{d\tau\mu}\phi_{a,r}^{R}\phi_{a,r-\hat{d\tau\tau}}^{I} - e^{d\tau\mu} \phi_{a,r}^{I}\phi_{a,r-\hat{\tau}}^{R}  \right)\\
& & -  \frac{1}{2} \sum_{a,b=1}^{2} \epsilon_{ab} \left(  e^{d\tau\mu} \phi_{a,r}^{R} \phi_{b,r-\hat{\tau}}^{R} - e^{d\tau\mu} \phi_{a,r}^{I} \phi_{b,r-\hat{\tau}}^{I} \right) \nonumber,
\eea
%
while the spatial derivative piece $S_{\del,r}$ (from the kinetic energy) becomes:
%
\bea
S_{\del, r}^{R} & \rightarrow & \sum_{a=1}^{2}\left[\frac{d}{m}(\phi_{a,r}^{R})^{2}-\frac{d}{m} (\phi_{a,r}^{I})^{2} - \frac{1}{4m}\sum_{i = \pm x, y} \left(  \phi_{a,r}^{R} \phi_{a,r+\hat{i}}^{R} -  \phi_{a,r}^{I} \phi_{a,r+\hat{i}}^{I} \right)\right] \\
& & +  \frac{1}{4m}\sum_{a,b = 1}^{2} \sum_{i = \pm x,y}\epsilon_{ab}\left(  \phi_{a,r}^{R} \phi_{b,r+\hat{i}}^{I} +  \phi_{a,r}^{I} \phi_{b,r+\hat{i}}^{R} \right)\nonumber \\
S_{\del, r}^{I}& \rightarrow & \sum_{a=1}^{2}\left[ \frac{2d}{m}\phi_{a,r}^{R}\phi_{a,r}^{I} -\frac{1}{4m}\sum_{i = \pm x,y} \left( \phi_{a,r}^{R} \phi_{a,r+\hat{i}}^{I} +  \phi_{a,r}^{I} \phi_{a,r+\hat{i}}^{R}\right) \right]\\
& & -\frac{1}{4m}\sum_{a,b=1}^{2}\sum_{i = \pm x,y} \epsilon_{ab} \left(  \phi_{a,r}^{R} \phi_{b,r+\hat{i}}^{R} -  \phi_{a,r}^{I} \phi_{b,r+\hat{i}}^{I} \right)  \nonumber.
\eea
%
The trapping potential term, $S_{\text{trap}}$, becomes:
%
\bea
S_{\text{trap}, r}^{R} & \rightarrow &  \frac{m}{4}\omega_{\text{trap}}^{2}\left(x^{2} + y^{2}\right)\sum_{a=1}^{2}\left[\phi_{a,r}^{R}\phi_{a,r-\hat{\tau}}^{R} - \phi_{a,r}^{I}\phi_{a,r-\hat{\tau}}^{I} - \sum_{b=1}^{2}\epsilon_{ab}\left(\phi_{a,r}^{R}\phi_{b,r-\hat{\tau}}^{I}+\phi_{a,r}^{I}\phi_{b,r-\hat{\tau}}^{R} \right) \right]  \\
S_{\text{trap}, r}^{I} & \rightarrow & \frac{m}{4}\omega_{\text{trap}}^{2}\left(x^{2} + y^{2}\right)\sum_{a=1}^{2}\left[\phi_{a,r}^{R}\phi_{a,r-\hat{\tau}}^{I} + \phi_{a,r}^{I}\phi_{a,r-\hat{\tau}}^{R} + \sum_{b=1}^{2}\epsilon_{ab}\left(\phi_{a,r}^{R}\phi_{b,r-\hat{\tau}}^{R} - \phi_{a,r}^{I}\phi_{b,r-\hat{\tau}}^{I}\right)\right].
\eea
%
The rotating term, $S_{\omega,r}$ becomes:
%
\bea
S_{\omega, r}^{R} & \rightarrow & \frac{\omega_{z}}{2} \sum_{a = 1}^{2}(y-x)  \left[( \phi_{a,r}^{I}\phi_{a,r-\hat{\tau}}^{R} + \phi_{a,r}^{R}\phi_{a,r-\hat{\tau}}^{I} ) + \sum_{b=1}^{2}\epsilon_{ab} (\phi_{a,r}^{R}\phi_{b,r-\hat{\tau}}^{R} -  \phi_{a,r}^{I}\phi_{b,r-\hat{\tau}}^{I})\right]\nonumber \\
%
&&+  \frac{\omega_{z}}{2} \sum_{a= 1}^{2} \widetilde{x}\left[( \phi_{a,r}^{I}\phi_{a,r-\hat{y}-\hat{\tau}}^{R} + \phi_{a,r}^{R}\phi_{a,r-\hat{y}-\hat{\tau}}^{I} ) + \sum_{b=1}^{2}\epsilon_{ab} (\phi_{a,r}^{R}\phi_{b,r-\hat{y}-\hat{\tau}}^{R} - \phi_{a,r}^{I} \phi_{b,r-\hat{y}-\hat{\tau}}^{I}) \right]\nonumber \\
%
&& - \frac{\omega_{z}}{2} \sum_{a = 1}^{2}\widetilde{y}\left[ (\phi_{a,r}^{I}\phi_{a,r-\hat{x}-\hat{\tau}}^{R} +  \phi_{a,r}^{R}\phi_{a,r-\hat{x}-\hat{\tau}}^{I}) + \sum_{b=1}^{2}\epsilon_{ab} (\phi_{a,r}^{R}\phi_{b,r-\hat{x}-\hat{\tau}}^{R}   - \phi_{a,r}^{I} \phi_{b,r-\hat{x}-\hat{\tau}}^{I}) \right]  \\
%
S_{\omega, r}^{I}& \rightarrow &  \frac{\omega_{z}}{2}\sum_{a = 1}^{2}(x-y) \left[ (\phi_{a,r}^{R}\phi_{a,r-\hat{\tau}}^{R} - \phi_{a,r}^{I}\phi_{a,r-\hat{\tau}}^{I}) - \sum_{b=1}^{2}\epsilon_{ab} (\phi_{a,r}^{I}\phi_{b,r-\hat{\tau}}^{R}+ \phi_{a,r}^{R}\phi_{b,r-\hat{\tau}}^{I})\right] \nonumber\\
% 
&& -\frac{\omega_{z}}{2}\sum_{a = 1}^{2} \widetilde{x}\left[ (\phi_{a,r}^{R}\phi_{a,r-\hat{y}-\hat{\tau}}^{R} - \phi_{a,r}^{I}\phi_{a,r-\hat{y}-\hat{\tau}}^{I}) - \sum_{b=1}^{2}\epsilon_{ab} ( \phi_{a,r}^{I}\phi_{b,r-\hat{y}-\hat{\tau}}^{R}+\phi_{a,r}^{R} \phi_{b,r-\hat{y}-\hat{\tau}}^{I} ) \right]\nonumber \\
%
&&+\frac{\omega_{z}}{2}\sum_{a = 1}^{2} \widetilde{y}\left[ (\phi_{a,r}^{R}\phi_{a,r-\hat{x}-\hat{\tau}}^{R}  - \phi_{a,r}^{I}\phi_{a,r-\hat{x}-\hat{\tau}}^{I})  - \sum_{b=1}^{2} \epsilon_{ab} (\phi_{a,r}^{I}\phi_{b,r-\hat{x}-\hat{\tau}}^{R} +  \phi_{a,r}^{R} \phi_{b,r-\hat{x}-\hat{\tau}}^{I}) \right]
\eea
%
and finally, the interaction term $S_{\text{int},r}$ becomes:
%
\bea
&&\text{**still to do: copy over these finished calculations from notebook} \nonumber\\
\text{Re}[S_{\text{int}, r}]  & \rightarrow &  \frac{\lambda}{4}  \sum_{a,b=1}^{2} \left[ 
%
\left( \phi_{a,r}^{R} \phi_{b,r-\hat{\tau}}^{I}\right)^{2} - \left( \phi_{a,r}^{I} \phi_{b,r-\hat{\tau}}^{I}\right)^{2} \right]\nonumber \\
%
&&+\frac{\lambda}{4}  \sum_{a,b=1}^{2} \left[ \phi_{a,r}^{R}\phi_{b,r}^{I}\left((\phi_{a,r-\hat{\tau}}^{R})^{2} - (\phi_{a,r-\hat{\tau}}^{I})^{2}\right) +\phi_{a,r}^{I}\phi_{b,r}^{R}\left((\phi_{a,r-\hat{\tau}}^{R})^{2} - (\phi_{a,r-\hat{\tau}}^{I})^{2} \right) \right]\nonumber \\
%
&&+\frac{\lambda}{4}  \sum_{a,b=1}^{2} \left[ \phi_{a,r-\hat{\tau}}^{I}\phi_{b,r-\hat{\tau}}^{R}\left((\phi_{a,r}^{I})^{2}-(\phi_{a,r}^{R})^{2} \right)+ \phi_{a, r-\hat{\tau}}^{R}\phi_{b,r-\hat{\tau}}^{I}\left((\phi_{a,r}^{I})^{2} - (\phi_{a,r}^{R})^{2}\right)\right]\nonumber \\
%
&&+\frac{\lambda}{4}  \sum_{a,b=1}^{2} 2 \left[\phi_{a,r}^{I}\phi_{b,r}^{I}\left(\phi_{a,r-\hat{\tau}}^{I}\phi_{b,r-\hat{\tau}}^{I}-\phi_{a,r-\hat{\tau}}^{R}\phi_{b,r-\hat{\tau}}^{R}\right) + \phi_{a,r}^{R}\phi_{b,r}^{R}\left(\phi_{a,r-\hat{\tau}}^{R}\phi_{b,r-\hat{\tau}}^{R}-\phi_{a,r-\hat{\tau}}^{I}\phi_{b,r-\hat{\tau}}^{I}\right)\right]\nonumber \\
%
&&-\frac{\lambda}{4}  \sum_{a,b=1}^{2} 2 \left[ \phi_{a,r}^{I} \phi_{b,r}^{R}\left( \phi_{a,r-\hat{\tau}}^{R}\phi_{b,r-\hat{\tau}}^{I}+\phi_{a,r-\hat{\tau}}^{I}\phi_{b,r-\hat{\tau}}^{R}\right)+ \phi_{a,r}^{R} \phi_{b,r}^{I}\left( \phi_{a,r-\hat{\tau}}^{R}\phi_{b,r-\hat{\tau}}^{I}+\phi_{a,r-\hat{\tau}}^{I}\phi_{b,r-\hat{\tau}}^{R}\right)\right]\nonumber \\
%
&&+\frac{\lambda}{4}  \sum_{a,b=1}^{2} 2 \left[\phi_{a,r}^{I}\phi_{a,r}^{R}\left( \phi_{a,r-\hat{\tau}}^{I}\phi_{b,r-\hat{\tau}}^{I}-\phi_{a,r-\hat{\tau}}^{R}\phi_{b,r-\hat{\tau}}^{R}\right) + \phi_{a,r-\hat{\tau}}^{I}\phi_{a,r-\hat{\tau}}^{R}\left(\phi_{a,r}^{R}\phi_{b,r}^{R} -\phi_{a,r}^{I}\phi_{b,r}^{I} \right)\right]\nonumber \\
%
&&+\frac{\lambda}{4}  \sum_{a,b=1}^{2} 4 \left[\phi_{a,r}^{I}\phi_{a,r}^{R}\phi_{b,r-\hat{\tau}}^{R}\phi_{b,r-\hat{\tau}}^{I} \right]\nonumber \\
%
%&&\text{Check the real part and do the imaginary part} \nonumber
%S_{\text{int}, r}^{R}  & \rightarrow & \frac{\lambda}{4} \sum_{a,b=1}^{2}\left[ (\phi_{a,r}^{R})^{2}(\phi_{b,r}^{R})^{2} - 2(\phi_{a,r}^{R})^{2}(\phi_{b,r}^{I})^{2}+ (\phi_{a,r}^{I})^{2}(\phi_{b,r}^{I})^{2}\right] - \lambda \sum_{a,b=1}^{2} \phi_{a,r}^{R}\phi_{a,r}^{I}\phi_{b,r}^{R} \phi_{b,r}^{I}\\
%S_{\text{int}, r}^{I} & \rightarrow &\frac{ \lambda}{2} \sum_{a,b=1}^{2} \left[(\phi_{a,r}^{R})^{2}\phi_{b,r}^{R}\phi_{b,r}^{I} - (\phi_{a,r}^{I})^{2}\phi_{b,r}^{R}\phi_{b,r}^{I}  \right]+\frac{ \lambda}{2} \sum_{a,b=1}^{2}  \left[  \phi_{a,r}^{R}\phi_{a,r}^{I}(\phi_{b,r}^{R})^{2}-\phi_{a,r}^{R}\phi_{a,r}^{I}(\phi_{b,r}^{I})^{2}\right],
\eea 
%
where in all of the above, $S_{j} = S_{j}^{R}+ i S_{j}^{I}$, and $\epsilon_{12} =1$, $\epsilon_{21} =-1$, and  $\epsilon_{11} = \epsilon_{22} = 0$. Note that we were able to compress the real part of the interaction due to the sum over $a$ and $b$. 

%Question - doesn't the imaginary part of the interaction need to be zero by definition? Or maybe I am misunderstanding how this works...


%%%%%%%%%%%%%%%%%%%%%%%%%%%%%%%%%%%%%%%%%%%%%%%%%%%%%%%%%%
%%%%%%%%%%%%%%%%%%% GENERATING CL EQUATIONS %%%%%%%%%%%%%%%%%%%%%%
%%%%%%%%%%%%%%%%%%%%%%%%%%%%%%%%%%%%%%%%%%%%%%%%%%%%%%%%%%
\section{\label{NRRBCLM}Generating the NRRB CL Equations}

\subsection{\label{LatticeDerivatives} Derivatives on the Lattice}
When taking derivatives of this lattice action with respect to the fields, we do the following:

%
\bea
\frac{\delta}{\delta \phi_{c,r}}\left(\sum_{q=1}^{N_{r}}\sum_{a=1}^{2}\phi_{a,q}\phi_{a,q+\hat{i}}\right)& =& \sum_{a=1}^{2}\sum_{q=1}^{N_{r}}\left( \phi_{a,q} \frac{\delta}{\delta  \phi_{c,r}} \phi_{a,q+\hat{i}} +\frac{\delta  \phi_{a,q}}{\delta  \phi_{c,r}}  \phi_{a,q+\hat{i}} \right)\\
&= & \sum_{a=1}^{2}\sum_{q=1}^{N_{r}}\left( \phi_{a,q}\delta_{c,a} \delta_{r,q+\hat{i}} +\delta_{c,a} \delta_{q,r} \phi_{a,q+\hat{i}} \right) \nonumber\\
&= &  \phi_{c,r-\hat{i}} +\phi_{c,r+\hat{i}} . \nonumber
\eea
%
Similarly, 
%
\beq
\frac{\delta}{\delta \phi_{c,r}}\left(\sum_{q=1}^{N_{r}}\sum_{a=1}^{2}\sum_{b=1}^{2}\epsilon_{ab}\phi_{a,q}\phi_{b,q+\hat{i}}\right) = \sum_{b=1}^{2}\epsilon_{cb}\left( \phi_{b,r-\hat{i}} +\phi_{b,r+\hat{i}} \right).
\eeq
%

\subsection{\label{KAlgebra}Computing the derivative of the action with respect to the real fields}
The first step in computing the CL Equations is to find $\frac{\delta S_{r}}{\delta \phi_{a,r}}$. This is done below, with the sum over $a = 1,2$ implied:
%
\bea
\frac{\delta S_{r}}{\delta \phi_{a,r}}  & =& \frac{\delta S_{\tau,r}}{\delta \phi_{a,r}} +  \frac{\delta S_{\del,r}}{\delta \phi_{a,r}} -\frac{\delta S_{\text{trap},r} }{\delta \phi_{a,r}} - \frac{\delta S_{\omega,r}}{\delta \phi_{a,r}}  + \frac{\delta S_{\text{int},r}}{\delta \phi_{a,r}} \\
&& \nonumber
\eea
%
Again, we proceed by modifying each of the 5 parts of the action. First, the time and chemical potential term:
%
\bea
2 \frac{\delta}{\delta \phi_{a,r}} S_{\tau,r} & =&  \frac{\delta}{\delta \phi_{a,r}}\sum_{a=1}^{2}\left[\phi_{a,r}^{2} - e^{ d\tau \mu}\phi_{a,r}\phi_{a,r - \hat{\tau}} - i e^{ d\tau \mu} \sum_{b=1}^{2}\epsilon_{ab}  \phi_{a,r}\phi_{b,r - \hat{\tau}} \right]\nonumber\\
2 \frac{\delta}{\delta \phi_{a,r}} S_{\tau,r}  & =& 2\phi_{a,r}- e^{ d\tau \mu}\left(\phi_{a,r - \hat{\tau}}+\phi_{a,r + \hat{\tau}}\right) - i e^{ d\tau \mu}\epsilon_{ab}\left(\phi_{b,r - \hat{\tau}} +\phi_{b,r + \hat{\tau}}\right)
\eea
%
Next, the spatial derivative part:
%
\bea
-\frac{4 m}{d\tau} \frac{\delta}{\delta \phi_{a,r}} S_{\del,r} & =& \frac{\delta}{\delta \phi_{a,r}}\sum_{a=1}^{2}\left[ \sum_{i = \pm x,y}\phi_{a,r}\phi_{a,r + \hat{i}} -2\phi_{a,r}^{2}+i\sum_{b=1}^{2}\sum_{i = \pm x,y}\epsilon_{ab} \phi_{a,r}\phi_{b,r + \hat{i}} \right] \nonumber\\
& =& \sum_{i = \pm x,y}\left(\phi_{a,r + \hat{i}} + \phi_{a,r - \hat{i}}\right) -4\phi_{a,r}\nonumber\\%+i \sum_{i = \pm x,y}\epsilon_{ab}\left(\phi_{b,r + \hat{i}}+\phi_{b,r - \hat{i}}\right)\nonumber\\
& =& 2 \sum_{i = \pm x,y}\phi_{a,r + \hat{i}} -4\phi_{a,r}\nonumber\\%+2 i  \sum_{i = \pm x,y}\epsilon_{ab}\phi_{b,r + \hat{i}}\nonumber\\
 -\frac{2 m}{d\tau} \frac{\delta}{\delta \phi_{a,r}} S_{\del,r}& =& \sum_{i = \pm x,y}\phi_{a,r + \hat{i}} - 2\phi_{a,r}%+ i  \sum_{i = \pm x,y}\epsilon_{ab}\phi_{b,r + \hat{i}}.
\eea
%
%the imaginary part cancels out when taking the derivative - do this with a = 1,2 and b=1,2 written out explicitly to see it
Then the part of the action due to the external trapping potential:
%
\bea
\frac{4}{d\tau m \omega_{\text{tr}}^{2}(r_{\perp}^{2})}\frac{\delta}{\delta \phi_{a,r}}  S_{\text{trap},r}& =&\frac{\delta}{\delta \phi_{a,r}}\sum_{a=1}^{2}\left( \phi_{a,r}\phi_{a,r-\hat{\tau}} + i \sum_{b=1}^{2}\epsilon_{ab}  \phi_{a,r}\phi_{b,r-\hat{\tau}}\right) \nonumber \\
& =&\sum_{a=1}^{2}\left( \phi_{a,r+\hat{\tau}}+\phi_{a,r-\hat{\tau}} + i \sum_{b=1}^{2}\epsilon_{ab} \left( \phi_{b,r+\hat{\tau}}+\phi_{b,r-\hat{\tau}}\right)\right) \nonumber \\
\frac{4}{d\tau m \omega_{\text{tr}}^{2}(r_{\perp}^{2})}\frac{\delta}{\delta \phi_{a,r}}  S_{\text{trap},r}& =&  \sum_{a=1}^{2}\left( (\phi_{a,r+\hat{\tau}}+\phi_{a,r-\hat{\tau}}) + i \sum_{b=1}^{2}\epsilon_{ab} \left( \phi_{b,r+\hat{\tau}}+\phi_{b,r-\hat{\tau}}\right) \right)
\eea
%
%updated 2.15.19 
where $r_{\perp}^{2} = x^{2} + y^{2}$. Next, the rotational piece:
%
\bea
\frac{2}{d\tau\omega_{z}}\frac{\delta}{\delta \phi_{a,r}}  S_{\omega,r}& =&\frac{\delta}{\delta \phi_{a,r}}\sum_{a=1}^{2} \left[\sum_{b=1}^{2}\epsilon_{ab}\left( x \phi_{a,r}\phi_{b,r - \hat{y}}  -y \phi_{a,r}\phi_{b,r - \hat{x}}\right)+ i \left( (x-y) \phi_{a,r}^{2}  - x \phi_{a,r}\phi_{a,r - \hat{y}}  + y \phi_{a,r}\phi_{a,r - \hat{x}} \right) \right]  \nonumber\\
%
& =&\epsilon_{ab} \left[x \left(\phi_{b,r - \hat{y}} + \phi_{b,r + \hat{y}}\right)-y \left(\phi_{b,r - \hat{x}} + \phi_{b,r + \hat{x}}\right) \right] \nonumber \\
& & + i \left[ 2(x-y) \phi_{a,r} - x \left(\phi_{a,r - \hat{y} }+ \phi_{a,r + \hat{y}}\right) + y \left(\phi_{a,r - \hat{x}}+\phi_{a,r + \hat{x}} \right) \right]    \nonumber \\
%
\frac{2}{d\tau\omega_{z}}\frac{\delta}{\delta \phi_{a,r}}  S_{\omega,r}& =&\epsilon_{ab} \left[x \left(\phi_{b,r - \hat{y}}+\phi_{b,r + \hat{y}}\right)-y \left(\phi_{b,r - \hat{x}}+\phi_{b,r + \hat{x}}\right) \right] \nonumber \\
& & + i \left[ 2(x-y) \phi_{a,r} - x \left(\phi_{a,r - \hat{y} }+ \phi_{a,r + \hat{y}}\right) + y \left(\phi_{a,r - \hat{x}}+\phi_{a,r + \hat{x}} \right) \right]    
\eea
%
And finally, the interaction term in the action:
%
\bea
\frac{4}{ d\tau\lambda}\frac{\delta}{\delta \phi_{a,r}}  S_{\text{int},r}& =& \frac{\delta}{\delta \phi_{a,r}}\sum_{a=1}^{2}\sum_{b=1}^{2}\phi_{a,r}^{2}\phi_{b,r}^{2} \nonumber\\
& =& 4 \phi_{a,r} \sum_{b=1}^{2}\phi_{b,r}^{2} \nonumber\\
\frac{1}{ d\tau\lambda}\frac{\delta}{\delta \phi_{a,r}}  S_{\text{int},r}& =& \phi_{a,r} \sum_{b=1}^{2}\phi_{b,r}^{2}.
\eea
%
Here, there is an extra factor of 2 that appears when solving this equation without the summation notation - I'm not really sure why it doesn't appear when using summation, but it is the correct factor.
 %%%%%%%%%%%%%%%%%%%%%%%%%%%%%%%%%%%%%%%%%%%%%
 %%%%%%%%%%%%%%%%%%%%%%%%%%%%%%%%%%%%%%%%%%%%%
 %%%%%%%%%%%%%%%%%%%%%%%%%%%%%%%%%%%%%%%%%%%%%
 
 \subsection{\label{2ndComplexification} Second Complexification of Drift Function}
 The next step is to complexify our real fields, $a$ and $b$, such that $\phi_{a} = \phi_{a}^{R} + i \phi_{a}^{I}$. We do this for each part of the drift function, $K_{a,r} = \frac{\delta S_{r}}{\delta \phi_{a,r}}$.

First, the time and chemical potential term:
%
\bea
2 \frac{\delta}{\delta \phi_{a,r}} S_{\tau,r}  & =& 2\phi_{a,r}- e^{ d\tau \mu}\left(\phi_{a,r - \hat{\tau}}+\phi_{a,r + \hat{\tau}}\right) - i e^{ d\tau \mu}\epsilon_{ab}\left(\phi_{b,r - \hat{\tau}} +\phi_{b,r + \hat{\tau}}\right)\nonumber \\
& =& 2(\phi_{a,r}^{R} + i \phi_{a,r}^{I})- e^{ d\tau \mu}\left(\phi_{a,r - \hat{\tau}}^{R}+i\phi_{a,r - \hat{\tau}}^{I}+\phi_{a,r + \hat{\tau}}^{R}+i\phi_{a,r + \hat{\tau}}^{I}\right) \nonumber \\
&&- i e^{ d\tau \mu}\epsilon_{ab}\left(\phi_{b,r - \hat{\tau}}^{R}+i\phi_{b,r - \hat{\tau}}^{I} +\phi_{b,r + \hat{\tau}}^{R}+i\phi_{b,r + \hat{\tau}}^{I}\right)\nonumber\\
& =& 2\phi_{a,r}^{R} - e^{ d\tau \mu}\left(\phi_{a,r - \hat{\tau}}^{R} +\phi_{a,r + \hat{\tau}}^{R}\right)+e^{d\tau \mu}\epsilon_{ab}\left(\phi_{b,r - \hat{\tau}}^{I}-\phi_{b,r + \hat{\tau}}^{I} \right)  \nonumber \\
&&+i  \left[2\phi_{a,r}^{I}- e^{ d\tau \mu}\left(\phi_{a,r - \hat{\tau}}^{I}+\phi_{a,r + \hat{\tau}}^{I}\right)-e^{d\tau \mu}\epsilon_{ab}\left(\phi_{b,r - \hat{\tau}}^{R}-\phi_{b,r + \hat{\tau}}^{R}\right)   \right] \nonumber
\eea
%
So
%
\bea
\text{Re}\left[ \frac{\delta}{\delta \phi_{a,r}} S_{\tau,r} \right]& =&  \phi_{a,r}^{R} - \frac{e^{ d\tau \mu}}{2}\left(\phi_{a,r - \hat{\tau}}^{R} +\phi_{a,r + \hat{\tau}}^{R}\right)+\frac{e^{ d\tau \mu}}{2}\epsilon_{ab}\left(\phi_{b,r - \hat{\tau}}^{I}-\phi_{b,r + \hat{\tau}}^{I} \right)  \\
\text{Im}\left[ \frac{\delta}{\delta \phi_{a,r}} S_{\tau,r} \right]&=& \phi_{a,r}^{I}- \frac{e^{ d\tau \mu}}{2}\left(\phi_{a,r - \hat{\tau}}^{I}+\phi_{a,r + \hat{\tau}}^{I}\right)-\frac{e^{ d\tau \mu}}{2}\epsilon_{ab}\left(\phi_{b,r - \hat{\tau}}^{R}-\phi_{b,r + \hat{\tau}}^{R}\right)
\eea
%
%This value checked in the Langevin evolution code (v2.2) on Oct 14, 2018
%checked in v3.5 on Jan 31, 2019
Next, the spatial derivative part:
%
\bea
 -\frac{2 m}{d\tau} \frac{\delta}{\delta \phi_{a,r}} S_{\del,r}& =& \sum_{i = \pm1}^{d}\phi_{a,r + \hat{i}} - 2d \phi_{a,r} \nonumber \\
 & =& \sum_{i =  \pm1}^{d}\left(\phi_{a,r + \hat{i}}^{R} + i  \phi_{a,r + \hat{i}}^{I}\right)- 2 d \phi_{a,r}^{R} - 2 i d  \phi_{a,r}^{I}\nonumber\\%+ i  \sum_{i = \pm x,y}\epsilon_{ab}\left(\phi_{b,r + \hat{i}}^{R}+i\phi_{b,r + \hat{i}}^{I}\right)\nonumber\\
 %& =& \sum_{i = \pm x,y}\left(\phi_{a,r + \hat{i}}^{R} - \epsilon_{ab}\phi_{b,r + \hat{i}}^{I}\right)- 2\phi_{a,r}^{R}+ i \left[  \sum_{i = \pm x,y}\left(\phi_{a,r + \hat{i}}^{I}+\epsilon_{ab}\phi_{b,r + \hat{i}}^{R}\right)- 2  \phi_{a,r}^{I} \right] \nonumber
  & =& \sum_{i = \pm1}^{d}\phi_{a,r + \hat{i}}^{R}- 2d \phi_{a,r}^{R} + i \left( \sum_{i = \pm1}^{d}\phi_{a,r + \hat{i}}^{I} - 2 d \phi_{a,r}^{I}\right)\nonumber
\eea
%
So
%
\bea
\text{Re}\left[\frac{\delta}{\delta \phi_{a,r}} S_{\del,r}\right] & =&\frac{d\tau}{2m} \left( 2 d \phi_{a,r}^{R}-\sum_{i = \pm1}^{d} \phi_{a,r + \hat{i}}^{R} \right) \\
\text{Im}\left[ \frac{\delta}{\delta \phi_{a,r}} S_{\del,r}\right] & =&\frac{d\tau}{2m} \left(2 d \phi_{a,r}^{I}-\sum_{i = \pm1}^{d}\phi_{a,r + \hat{i}}^{I} \right)
\eea
%
%This value updated in the Langevin evolution code (v2.2) on Oct 14, 2018
%updated again on Jan 31, 2019
Then the part of the action due to the external trapping potential:
%
\bea
\frac{4}{d\tau m \omega_{\text{tr}}^{2}(r_{\perp}^{2})}\frac{\delta}{\delta \phi_{a,r}}  S_{\text{trap},r}  &=&  \sum_{a=1}^{2}\left( (\phi_{a,r+\hat{\tau}}+\phi_{a,r-\hat{\tau}}) + i \sum_{b=1}^{2}\epsilon_{ab} \left( \phi_{b,r+\hat{\tau}}+\phi_{b,r-\hat{\tau}}\right) \right)\nonumber \\
& =& \sum_{a=1}^{2}\left( (\phi_{a,r+\hat{\tau}}^{R} +\phi_{a,r-\hat{\tau}}^{R} ) - \sum_{b=1}^{2}\epsilon_{ab} \left( \phi_{b,r+\hat{\tau}}^{I}+ \phi_{b,r-\hat{\tau}}^{I}\right) \right) \nonumber \\
& & +   i \sum_{a=1}^{2}\left( (\phi_{a,r+\hat{\tau}}^{I}  +  \phi_{a,r-\hat{\tau}}^{I} )+ \sum_{b=1}^{2}\epsilon_{ab}(\phi_{b,r+\hat{\tau}}^{R}+\phi_{b,r-\hat{\tau}}^{R}  )\right)
\eea
%
So
%
\bea
\text{Re}\left[\frac{\delta}{\delta \phi_{a,r}}  S_{\text{trap},r}\right] & =&\frac{d\tau\omega_{\text{tr}}^{2}(r_{\perp}^{2})}{4} \sum_{a=1}^{2}\left( (\phi_{a,r+\hat{\tau}}^{R} +\phi_{a,r-\hat{\tau}}^{R} ) - \sum_{b=1}^{2}\epsilon_{ab} \left( \phi_{b,r+\hat{\tau}}^{I}+ \phi_{b,r-\hat{\tau}}^{I}\right) \right)  \\
\text{Im}\left[\frac{\delta}{\delta \phi_{a,r}}  S_{\text{trap},r}\right] & =&\frac{d\tau\omega_{\text{tr}}^{2}(r_{\perp}^{2})}{4}   \sum_{a=1}^{2}\left( (\phi_{a,r+\hat{\tau}}^{I}  +  \phi_{a,r-\hat{\tau}}^{I} )+ \sum_{b=1}^{2}\epsilon_{ab}(\phi_{b,r+\hat{\tau}}^{R}+\phi_{b,r-\hat{\tau}}^{R}  )\right)
\eea
%
where $r_{\perp}^{2} = x^{2} + y^{2}$. Next, the rotational piece:

%
\bea
\frac{2}{d\tau\omega_{z}}\frac{\delta}{\delta \phi_{a,r}}  S_{\omega,r}& =&\epsilon_{ab} \left[x \left(\phi_{b,r - \hat{y}}+\phi_{b,r + \hat{y}}\right)-y \left(\phi_{b,r - \hat{x}}+\phi_{b,r + \hat{x}}\right) \right] \nonumber \\
& & + i \left[ 2(x-y) \phi_{a,r} - x \left(\phi_{a,r - \hat{y} }+ \phi_{a,r + \hat{y}}\right) + y \left(\phi_{a,r - \hat{x}}+\phi_{a,r + \hat{x}} \right) \right]   \nonumber \\
%
& = &\epsilon_{ab} \left[x \left(\phi_{b,r - \hat{y}}^{R} + i \phi_{b,r - \hat{y}}^{I} +\phi_{b,r + \hat{y}}^{R} + i \phi_{b,r + \hat{y}}^{I} \right)-y \left(\phi_{b,r - \hat{x}}^{R}+i\phi_{b,r - \hat{x}}^{I}+\phi_{b,r + \hat{x}}^{R} + i \phi_{b,r + \hat{x}}^{I}\right) \right] \nonumber \\
& & + i \left[ 2(x-y) \phi_{a,r}^{R} + 2i (x-y) \phi_{a,r}^{I} - x \left(\phi_{a,r - \hat{y} }^{R} + i \phi_{a,r - \hat{y} }^{I} + \phi_{a,r + \hat{y}}^{R} + i \phi_{a,r + \hat{y}}^{I}\right) \right] \nonumber \\
& & +i \left[ y \left(\phi_{a,r - \hat{x}}^{R} + i \phi_{a,r - \hat{x}}^{I}+\phi_{a,r + \hat{x}}^{R} + i \phi_{a,r + \hat{x}}^{I}\right) \right]   \nonumber \\
%
& = &- 2 (x-y) \phi_{a,r}^{I} + x \left( \phi_{a,r - \hat{y} }^{I} + \phi_{a,r + \hat{y}}^{I}\right) - y \left( \phi_{a,r - \hat{x}}^{I} + \phi_{a,r + \hat{x}}^{I}\right)\nonumber \\
&& + \epsilon_{ab} \left[x \left(\phi_{b,r - \hat{y}}^{R} +\phi_{b,r + \hat{y}}^{R}  \right)-y \left(\phi_{b,r - \hat{x}}^{R}+\phi_{b,r + \hat{x}}^{R}\right) \right] \nonumber \\
& & + i \left[ 2(x-y) \phi_{a,r}^{R}  - x \left(\phi_{a,r - \hat{y} }^{R}  + \phi_{a,r + \hat{y}}^{R} \right) + y \left(\phi_{a,r - \hat{x}}^{R} +\phi_{a,r + \hat{x}}^{R} \right) \right] \nonumber \\
& & + i \epsilon_{ab} \left[ x \left( \phi_{b,r - \hat{y}}^{I} + \phi_{b,r + \hat{y}}^{I}\right) - y \left(\phi_{b,r - \hat{x}}^{I} + \phi_{b,r + \hat{x}}^{I} \right)\right]  
%
\eea
%
%THIS OLD DERIVATION  IS PROBABLY WRONG
%
%\bea
%\frac{2}{d\tau\omega_{z}}\frac{\delta}{\delta \phi_{a,r}}  S_{\omega,r}& =&\epsilon_{ab} \left[x \left(\phi_{b,r - \hat{y}}-\phi_{b,r + \hat{y}}\right)-y \left(\phi_{b,r - \hat{x}}-\phi_{b,r + \hat{x}}\right) \right] \nonumber \\
%& & + i \left[ 2(x-y) \phi_{a,r} - x \left(\phi_{a,r - \hat{y} }+ \phi_{a,r + \hat{y}}\right) + y \left(\phi_{a,r - \hat{x}}+\phi_{a,r + \hat{x}} \right) \right]   \nonumber \\
%& =&\epsilon_{ab} \left[x \left(\phi_{b,r - \hat{y}}^{R}+ i \phi_{b,r - \hat{y}}^{I}-\phi_{b,r + \hat{y}}^{R} -i \phi_{b,r + \hat{y}}^{I}\right)-y \left(\phi_{b,r - \hat{x}}^{R} + i \phi_{b,r - \hat{x}}^{I}-\phi_{b,r + \hat{x}}^{R} - i \phi_{b,r + \hat{x}}^{I}\right) \right] \nonumber \\
%& & + i \left[ 2(x-y) \left(\phi_{a,r}^{R} + i \phi_{a,r} ^{I}\right) - x \left(\phi_{a,r - \hat{y} }^{R} + i \phi_{a,r - \hat{y} }^{I}+ \phi_{a,r + \hat{y}}^{R} + i \phi_{a,r + \hat{y}}^{I}\right) \right] \nonumber \\
%& & +i \left[ y \left(\phi_{a,r - \hat{x}}^{R} +i \phi_{a,r - \hat{x}}^{I}+\phi_{a,r + \hat{x}}^{R} + i \phi_{a,r + \hat{x}}^{I} \right) \right]   \nonumber \\
%& =& \epsilon_{ab} \left[ x \left(\phi_{b,r - \hat{y}}^{R}-\phi_{b,r + \hat{y}}^{R}\right) - y \left(\phi_{b,r - \hat{x}}^{R} -\phi_{b,r + \hat{x}}^{R}  \right)\right] - 2(x-y)\phi_{a,r} ^{I} \nonumber \\
%& & + x \left(\phi_{a,r - \hat{y} }^{I}+\phi_{a,r + \hat{y}}^{I}\right) - y \left(\phi_{a,r - \hat{x}}^{I}  +\phi_{a,r + \hat{x}}^{I} \right)\nonumber\\
%& & + i \left[ \epsilon_{ab} \left(  x( \phi_{b,r - \hat{y}}^{I} -  \phi_{b,r + \hat{y}}^{I}) - y (\phi_{b,r - \hat{x}}^{I}- \phi_{b,r + \hat{x}}^{I})\right)+2(x-y)\phi_{a,r}^{R} \right]\nonumber\\
%&& + i \left[y \left(\phi_{a,r - \hat{x}}^{R}+\phi_{a,r + \hat{x}}^{R}\right)- x\left( \phi_{a,r - \hat{y} }^{R} + \phi_{a,r + \hat{y}}^{R} \right) \right]\nonumber
%\eea
%
So
%
\bea
\text{Re}\left[\frac{\delta}{\delta \phi_{a,r}}  S_{\omega_{z},r}\right] & = & \frac{d\tau\omega_{z}}{2} \left[ x \left( \phi_{a,r - \hat{y} }^{I} + \phi_{a,r + \hat{y}}^{I}\right) - y \left( \phi_{a,r - \hat{x}}^{I} + \phi_{a,r + \hat{x}}^{I}\right) - 2 (x-y) \phi_{a,r}^{I} \right]\nonumber \\
&& +\frac{d\tau\omega_{z}}{2} \epsilon_{ab} \left[x \left(\phi_{b,r - \hat{y}}^{R} +\phi_{b,r + \hat{y}}^{R}  \right)-y \left(\phi_{b,r - \hat{x}}^{R}+\phi_{b,r + \hat{x}}^{R}\right) \right] \nonumber \\
%
\text{Im}\left[\frac{\delta}{\delta \phi_{a,r}}  S_{\omega_{z},r}\right] & =&\frac{d\tau\omega_{z}}{2} \left[ 2(x-y) \phi_{a,r}^{R}  - x \left(\phi_{a,r - \hat{y} }^{R}  + \phi_{a,r + \hat{y}}^{R} \right) + y \left(\phi_{a,r - \hat{x}}^{R} +\phi_{a,r + \hat{x}}^{R} \right)\right]  \\
&& + \frac{\omega_{z}}{2} \epsilon_{ab} \left[ x \left( \phi_{b,r - \hat{y}}^{I} + \phi_{b,r + \hat{y}}^{I}\right) - y \left(\phi_{b,r - \hat{x}}^{I} + \phi_{b,r + \hat{x}}^{I} \right)\right]\nonumber
\eea
%
%This value updated in the Langevin evolution code (v2.2) on Oct 14, 2018
%Updated again Jan 29, 2019
And finally, the interaction term in the action:
%
\bea
\frac{1}{d\tau \lambda}\frac{\delta}{\delta \phi_{a,r}}  S_{\text{int},r}& =& \phi_{a,r} \sum_{b=1}^{2}\phi_{b,r}^{2} \nonumber \\
& =& (\phi_{a,r}^{R} + i \phi_{a,r}^{I}) \sum_{b=1}^{2}(\phi_{b,r}^{R} + i \phi_{b,r}^{I})^{2} \nonumber \\
& =& \phi_{a,r}^{R}\sum_{b=1}^{2}\left((\phi_{b,r}^{R})^{2} +2 i \phi_{b,r}^{R}\phi_{b,r}^{I} - (\phi_{b,r}^{I})^{2}\right)+ i \phi_{a,r}^{I} \sum_{b=1}^{2}\left((\phi_{b,r}^{R})^{2} +2 i \phi_{b,r}^{R}\phi_{b,r}^{I} - (\phi_{b,r}^{I})^{2}\right)\nonumber \\
& =&\sum_{b=1}^{2}\left[ \phi_{a,r}^{R}(\phi_{b,r}^{R})^{2} +2 i \phi_{a,r}^{R}\phi_{b,r}^{R}\phi_{b,r}^{I} - \phi_{a,r}^{R}(\phi_{b,r}^{I})^{2}+ i \phi_{a,r}^{I}(\phi_{b,r}^{R})^{2} -2  \phi_{a,r}^{I}\phi_{b,r}^{R}\phi_{b,r}^{I} - i \phi_{a,r}^{I}(\phi_{b,r}^{I})^{2}\right]\nonumber
\\
& =&\sum_{b=1}^{2}\left[ \phi_{a,r}^{R}(\phi_{b,r}^{R})^{2} - \phi_{a,r}^{R}(\phi_{b,r}^{I})^{2} -2  \phi_{a,r}^{I}\phi_{b,r}^{R}\phi_{b,r}^{I} \right] + i \sum_{b=1}^{2}\left[ 2 \phi_{a,r}^{R}\phi_{b,r}^{R}\phi_{b,r}^{I} + \phi_{a,r}^{I}(\phi_{b,r}^{R})^{2} - \phi_{a,r}^{I}(\phi_{b,r}^{I})^{2}\right]\nonumber
\eea
%
So
%
\bea
\text{Re}\left[\frac{\delta}{\delta \phi_{a,r}} S_{\text{int},r}\right] & =& d\tau\lambda \sum_{b=1}^{2}\left[ \phi_{a,r}^{R}(\phi_{b,r}^{R})^{2} - \phi_{a,r}^{R}(\phi_{b,r}^{I})^{2} -2  \phi_{a,r}^{I}\phi_{b,r}^{R}\phi_{b,r}^{I} \right] \\
\text{Im}\left[\frac{\delta}{\delta \phi_{a,r}} S_{\text{int},r}\right] & =&d\tau\lambda \sum_{b=1}^{2}\left[ \phi_{a,r}^{I}(\phi_{b,r}^{R})^{2}+2 \phi_{a,r}^{R}\phi_{b,r}^{R}\phi_{b,r}^{I}  - \phi_{a,r}^{I}(\phi_{b,r}^{I})^{2}\right]
\eea
%
%This value updated in the Langevin evolution code (v2.2) on Oct 14, 2018

%\section{WHAT IS THIS?}

%When we do this, the derivative of the action becomes:
%\bea
%\text{Re}\left[\frac{\delta S}{\delta \phi_{a,r}}\right]  &=& (1 + \frac{2 d\tau}{m})\phi_{a,r}^{R}- \frac{e^{d\tau\mu}}{2}(\phi_{a,r+\hat{\tau}}^{R}+ \phi_{a,r-\hat{\tau}}^{R}) -\frac{d\tau}{2m} \sum_{i=\pm x,y}\phi_{a,r+\hat{i}}^{R}\\
%&& + d\tau \omega_{z} (x-y)\phi_{a,r}^{I} - \frac{d\tau\omega_{z}(x-N_{x}/2)}{2}\sum_{i = \pm y}\phi_{a,r+\hat{i}}^{I} + \frac{d\tau\omega_{z}(y-N_{x}/2)}{2}\sum_{i = \pm x}\phi_{a,r+\hat{i}}^{I} \nonumber \\
%&& -\sum_{b}\epsilon_{ab}\left( \frac{d\tau\omega_{z} (x-N_{x}/2)}{2} \phi_{b,r-\hat{y}}^{R}-\frac{d\tau\omega_{z} (y-N_{x}/2)}{2} \phi_{b,r-\hat{x}}^{R} + \frac{e^{d\tau\mu}}{2}\phi_{b,r-\hat{\tau}}^{R} \right) \nonumber\\
%&& +\sum_{b}\epsilon_{ab}\frac{d\tau}{4m}\sum_{i = \pm x,y}\phi_{b,r+\hat{i}}^{I} \nonumber\\
%&&+\sum_{b}\frac{d\tau\lambda}{2} \left[\phi_{a,r}^{R}(\phi_{b,r}^{R})^{2}- 2 \phi_{a,r}^{I}\phi_{b,r}^{R} \phi_{b,r}^{I} - \phi_{a,r}^{R}(\phi_{b,r}^{I} )^{2}  \right]\nonumber\\
%
%\text{Im}\left[\frac{\delta S}{\delta \phi_{a,r}}\right]  &=& (1 + \frac{2d\tau}{m})\phi_{a,r}^{I}- \frac{e^{d\tau\mu}}{2}(\phi_{a,r+\hat{\tau}}^{I}+ \phi_{a,r-\hat{\tau}}^{I}) -\frac{d\tau}{2m} \sum_{i=\pm x,y}\phi_{a,r+\hat{i}}^{I} \\
%&& -d\tau \omega_{z} (x-y)\phi_{a,r}^{R} + \frac{d\tau\omega_{z}(x-N_{x}/2)}{2}\sum_{i = \pm y}\phi_{a,r+\hat{i}}^{R} - \frac{d\tau\omega_{z}(y-N_{x}/2)}{2}\sum_{i = \pm x}\phi_{a,r+\hat{i}}^{R} \nonumber \\
%&& -\sum_{b}\epsilon_{ab}\left( \frac{d\tau\omega_{z} (x-N_{x}/2)}{2} \phi_{b,r-\hat{y}}^{I}-\frac{d\tau\omega_{z}( y-N_{x}/2)}{2} \phi_{b,r-\hat{x}}^{I} +\frac{e^{d\tau\mu}}{2}\phi_{b,r-\hat{\tau}}^{I} \right) \nonumber\\
%&& -  \sum_{b}\epsilon_{ab}\frac{d\tau}{4m}\sum_{i = \pm x,y}\phi_{b,r+\hat{i}}^{R}\nonumber\\
%&&+\sum_{b}\frac{d\tau\lambda}{2} \left[\phi_{a,r}^{I}(\phi_{b,r}^{R})^{2} + 2 \phi_{a,r}^{R}\phi_{b,r}^{R} \phi_{b,r}^{I} - \phi_{a,r}^{I}(\phi_{b,r}^{I} )^{2}  \right] \nonumber
%\eea


%%%%%%%%%%%%%%%%%%%%%%%%%%%%%%%%%%%%%%%%%%%%%
%%%%%%%%%%%%%%%%%% OBSERVABLES %%%%%%%%%%%%%%%%%%
%%%%%%%%%%%%%%%%%%%%%%%%%%%%%%%%%%%%%%%%%%%%%

\section{\label{NRRBObservables}Lattice Observables}
The observables:
\subsection{Average Density}
The average density is the sum over the local density at each spatial site, normalized by the spatial lattice volume:
%
\beq
\langle \hat{n} \rangle = \frac{1}{N_{x}^{d}} \sum_{r} n_{r} 
\eeq
%
where $n_{r}$ can be found by taking the derivative of the lattice action with respect to the chemical potential, and then complexifying the fields as we have done before:
%
\bea
n_{r}  & = &- \frac{\partial}{\partial \beta \mu} S_{\tau, r} = -\frac{\partial}{\partial \beta \mu} \left(\phi_{r}^{*}\phi_{r} - \phi_{r}^{*}e^{\beta \mu / N_{\tau}} \phi_{r-\hat{\tau}} \right) =  \frac{\partial}{\partial \beta \mu} \left(\phi_{r}^{*}e^{\beta \mu / N_{\tau}} \phi_{r-\hat{\tau}} \right) \nonumber \\
%
&=& \frac{1}{N_{\tau}}e^{\beta \mu / N_{\tau}}  \left(\phi_{r}^{*}\phi_{r-\hat{\tau}} \right) = \frac{1}{2 N_{\tau}}e^{\beta \mu / N_{\tau}}  \left(\phi_{1,r}- i\phi_{2,r} \right) \left(\phi_{1,r-\hat{\tau}}+i\phi_{2,r-\hat{\tau}} \right) \nonumber\\
%
&=& \nonumber  \frac{1}{N_{\tau}}e^{\beta \mu / N_{\tau}} \sum_{a=1}^{2}\left[ \phi_{a,r}\phi_{a,r-\hat{\tau}} + i \sum_{b=1}^{2}\epsilon_{ab}\phi_{a,r}\phi_{b,r-\hat{\tau}}\right]\\
%
&= &\frac{1}{2 N_{\tau}}e^{\beta \mu / N_{\tau}}\sum_{a=1}^{2}\left[\phi_{a,r}^{R}\phi_{a,r-\hat{\tau}}^{R}- \phi_{a,r}^{I} \phi_{a,r-\hat{\tau}}^{I}  - \sum_{b=1}^{2} \epsilon_{ab} (\phi_{a,r}^{I}\phi_{b,r-\hat{\tau}}^{R}  +  \phi_{a,r}^{R}\phi_{b,r-\hat{\tau}}^{I})\right] \nonumber \\
& + & \frac{i}{2 N_{\tau}}e^{\beta \mu / N_{\tau}}\sum_{a=1}^{2}\left[ \phi_{a,r-\hat{\tau}}^{I}\phi_{a,r-\hat{\tau}}^{R}+ \phi_{a,r-\hat{\tau}}^{R}\phi_{a,r-\hat{\tau}}^{I} + \sum_{b=1}^{2} \epsilon_{ab}(\phi_{a,r}^{R}\phi_{b,r-\hat{\tau}}^{R} -\phi_{a,r}^{I}\phi_{b,r-\hat{\tau}}^{I})  \right]
\eea
%

\subsection{Average Angular Momentum}
The angular momentum operator, $L_{z}$ can be written
\beq
L_{z} = ((x-x_{0}) p_{y} - (y-y_{0}) p_{x}) = - i \hbar ((x-x_{0})\partial_{y} - (y-y_{0}) \partial_{x}).
\eeq We are interested in the expectation value of this operator: $\langle L_{z} \rangle$, which is found by summing over the lattice:
\beq
\langle L_{z} \rangle =- i \sum_{r} \phi_{r}^{*}((x-\frac{N_{x}-1}{2})\partial_{y} - (y-\frac{N_{x}-1}{2}) \partial_{x})\phi_{r},
\eeq 
%
where $\hbar \rightarrow 1$. First, let us implement our lattice derivative: $\partial_{j}\phi_{r} = \frac{1}{a^{2}}(\phi_{r -\hat{j}} - \phi_{r})$ (recall that our lattice spacing is $a = 1$):
%
\bea
\langle L_{z} \rangle &=& -i \sum_{r} \phi_{r}^{*}((x-r_{c})\phi_{r-\hat{y}} - x \phi_{r} - (y-r_{c}) \phi_{r-\hat{x}}+y \phi_{r})  \\
&=&  i \sum_{r} \left((y-r_{c}) \phi_{r}^{*}\phi_{r-\hat{x}}  - (x-r_{c})  \phi_{r}^{*}\phi_{r-\hat{y}}- (y-x)  \phi_{r}^{*}\phi_{r}\right)\nonumber
\eea 
%
Here we have also written $\frac{N_{x}-1}{2}$ as $r_{c}$ for simplicity. Let us write $\phi$ as $\frac{1}{\sqrt{2}}\left(\phi_{1} + i \phi_{2}\right)$; then, our angular momentum operator becomes a sum over the lattice sites and the real fields $\phi_{1}$ and $\phi_{2}$:
%
\bea
\langle L_{z} \rangle &=&\frac{i}{2}\sum_{r} (y-r_{c}) \left(\phi_{1,r} \phi_{1,r-\hat{x}} + i \phi_{1,r} \phi_{2,r-\hat{x}}- i \phi_{2,r}\phi_{1,r-\hat{x}}+ \phi_{2,r}\phi_{2,r-\hat{x}}\right) \nonumber \\ 
& &  - \frac{i}{2}\sum_{r} (x-r_{c})\left(\phi_{1,r} \phi_{1,r-\hat{y}} + i \phi_{1,r} \phi_{2,r-\hat{y}}- i \phi_{2,r}\phi_{1,r-\hat{y}}+ \phi_{2,r}\phi_{2,r-\hat{y}}\right)  \\
& & - \frac{i}{2}\sum_{r}(y-x) ( \phi_{1,r}^{2}+\phi_{2,r}^{2}) \nonumber\\
& = &  \frac{1}{2}\sum_{r}\sum_{a=1}^{2} \left((y-r_{c})\phi_{a,r} \phi_{a,r-\hat{x}} -(x-r_{c})\phi_{a,r} \phi_{a,r-\hat{y}}  -(y-x)\phi_{a,r}^{2} \right)  \\
& & - \frac{i}{2}\sum_{r}\sum_{a,b=1}^{2} \epsilon_{ab}\left((x-r_{c}) \phi_{a,r} \phi_{b,r-\hat{y}} - (y-r_{c}) \phi_{a,r}\phi_{b,r-\hat{x}} \right)\nonumber
\eea 
%
Next, we must complexify our real fields: $\phi_{a} = \phi_{a}^{R} + i \phi_{a}^{I}$, leading us to the following equation for the angular momentum in terms of our four lattice fields:
%
\bea
\langle L_{z} \rangle &=& -\frac{i}{2}\sum_{r}\sum_{a=1}^{2}\left( (x-r_{c})(\phi_{a,r}^{R} \phi_{a,r-\hat{y}}^{R} - \phi_{a,r}^{I} \phi_{a,r-\hat{y}}^{I} ) -(y-r_{c})( \phi_{a,r}^{R} \phi_{a,r-\hat{x}}^{R} - \phi_{a,r}^{I} \phi_{a,r-\hat{x}}^{I} )  \right) \nonumber \\
%
&&  -\frac{i}{2}\sum_{r}\sum_{a=1}^{2}(y-x)\left((\phi_{a,r}^{R})^{2}-(\phi_{a,r}^{I})^{2}  \right)\nonumber \\
%
& & -  \frac{i}{2}\sum_{r}\sum_{a,b=1}^{2} \epsilon_{ab}\left((x-r_{c})( \phi_{a,r}^{R} \phi_{b,r-\hat{y}}^{I} +  \phi_{a,r}^{I} \phi_{b,r-\hat{y}}^{R}) - (y-r_{c})(\phi_{a,r}^{R} \phi_{b,r-\hat{x}}^{I} + \phi_{a,r}^{I} \phi_{b,r-\hat{x}}^{R})\right)\nonumber \\
%
&& +\frac{1}{2}\sum_{r}\sum_{a=1}^{2}\left(  (x-r_{c})( \phi_{a,r}^{R} \phi_{a,r-\hat{y}}^{I} + \phi_{a,r}^{I} \phi_{a,r-\hat{y}}^{R} )-(y-r_{c}) (\phi_{a,r}^{R} \phi_{a,r-\hat{x}}^{I} + \phi_{a,r}^{I} \phi_{a,r-\hat{x}}^{R} )\right) \nonumber\\
%
& & + \sum_{r}\sum_{a,b=1}^{2}(y-x)\phi_{a,r}^{R}\phi_{a,r}^{I}\nonumber \\
%
& & +  \frac{1}{2}\sum_{r}\sum_{a,b=1}^{2} \epsilon_{ab}\left((x-r_{c})( \phi_{a,r}^{R} \phi_{b,r-\hat{y}}^{R} -  \phi_{a,r}^{I} \phi_{b,r-\hat{y}}^{I}) - (y-r_{c})(\phi_{a,r}^{R} \phi_{b,r-\hat{x}}^{R} -  \phi_{a,r}^{I} \phi_{b,r-\hat{x}}^{I}  ) \right).\nonumber
%
\eea 
When this is divided into the real and imaginary parts of the observable, we get:
%
\bea
\text{Re}\langle L_{z} \rangle &=& \frac{1}{2}\sum_{r}\sum_{a=1}^{2}\left(  (x-r_{c})( \phi_{a,r}^{R} \phi_{a,r-\hat{y}}^{I} + \phi_{a,r}^{I} \phi_{a,r-\hat{y}}^{R} )-(y-r_{c}) (\phi_{a,r}^{R} \phi_{a,r-\hat{x}}^{I} + \phi_{a,r}^{I} \phi_{a,r-\hat{x}}^{R} )\right) \nonumber\\
%
& & + \sum_{r}\sum_{a=1}^{2}(y-x)\phi_{a,r}^{R}\phi_{a,r}^{I} \\
%
& & +  \frac{1}{2}\sum_{r}\sum_{a,b=1}^{2} \epsilon_{ab}\left((x-r_{c})( \phi_{a,r}^{R} \phi_{b,r-\hat{y}}^{R} -  \phi_{a,r}^{I} \phi_{b,r-\hat{y}}^{I}) - (y-r_{c})(\phi_{a,r}^{R} \phi_{b,r-\hat{x}}^{R} -  \phi_{a,r}^{I} \phi_{b,r-\hat{x}}^{I}  ) \right).\nonumber \\
%
\text{Im}\langle L_{z} \rangle &=&-\frac{1}{2}\sum_{r}\sum_{a=1}^{2}\left((x-r_{c})(\phi_{a,r}^{R} \phi_{a,r-\hat{y}}^{R} - \phi_{a,r}^{I} \phi_{a,r-\hat{y}}^{I} ) - (y-r_{c})( \phi_{a,r}^{R} \phi_{a,r-\hat{x}}^{R} - \phi_{a,r}^{I} \phi_{a,r-\hat{x}}^{I} )  \right) \nonumber \\
%
&&  -\frac{1}{2}\sum_{r}\sum_{a=1}^{2}(y-x)\left((\phi_{a,r}^{R})^{2}-(\phi_{a,r}^{I})^{2}  \right) \\
%
& & -  \frac{1}{2}\sum_{r}\sum_{a,b=1}^{2} \epsilon_{ab}\left((x-r_{c})( \phi_{a,r}^{R} \phi_{b,r-\hat{y}}^{I} +  \phi_{a,r}^{I} \phi_{b,r-\hat{y}}^{R}) - (y-r_{c})(\phi_{a,r}^{R} \phi_{b,r-\hat{x}}^{I} + \phi_{a,r}^{I} \phi_{b,r-\hat{x}}^{R})\right)\nonumber 
%
\eea 
%
\subsection{Circulation}
The circulation is defined as
\bea
\Gamma[l] &=& \frac{1}{2 \pi}\oint_{l \times l}dx \left(\theta_{t,x+j} - \theta_{t,x}\right) \\
\theta_{t,x} &=& \tan^{-1}\left(\frac{\text{Im}[\phi_{t,x}]}{\text{Re}[\phi_{t,x}]} \right).
\eea
Given that our field, $\phi$, is broken into 4 components ($\phi_{1/2}^{R/I}$, we need to rewrite this quantity in terms of those components:
\bea
\phi &=& \frac{1}{\sqrt{2}}\left(\phi_{1}^{R} + i \phi_{1}^{I} + i \phi_{2}^{R} - \phi_{2}^{I}\right)\\
\text{Re}[\phi_{t,x}] & =& \frac{1}{\sqrt{2}}\left(\phi_{1,t,x}^{R} - \phi_{2,t,x}^{I}\right)\\
\text{Im}[\phi_{t,x}] & = & \frac{1}{\sqrt{2}}\left(\phi_{1,t,x}^{I} + \phi_{2,t,x}^{R} \right)\\
\theta_{t,x} &=& \tan^{-1}\left(\frac{\phi_{1,t,x}^{I} + \phi_{2,t,x}^{R}}{\phi_{1,t,x}^{R} - \phi_{2,t,x}^{I}} \right).
\eea
Therefore, our circulation is a sum computed around a loop on the lattice of length $l$ in each direction:
\beq
\Gamma[l] = \frac{1}{2 \pi}\sum_{l \times l}\left(\theta_{t,x+j} - \theta_{t,x}\right) \\
\eeq
where $\theta_{t,x+j}$ is computed at the next site on the loop from $\theta_{t,x}$ for each point along the loop.


\subsection{Average Energy}
The expectation value of the density is defined as
\beq
\langle E \rangle = \frac{-\partial \ln \CZ}{\partial \beta}= \frac{-\partial \ln (e^{-S})}{\partial \beta} = \frac{\partial S}{\partial \beta}.
\eeq
Since we have written the action in terms of $\beta$ as 
%
\beq
S \rightarrow \frac{1}{N_{\tau}}\sum_{x,\tau}\Delta x^{d}\beta \left[\left(\partial_{\tau}  - \frac{1}{2m}\del^{2} - \mu- i \omega_{z}(x \partial_{y} - y \partial_{x})  - \frac{m\omega_{\text{trap}}^{2}}{2}(x^{2}+y^{2}) \right)\phi + \lambda (\phi^{*}\phi)^{2}\right].
\eeq 
%
We can see that the action is linear in $\beta$. Therefore, we can simply calculate the energy in the following way:
\beq
\langle E \rangle = \frac{S}{ \beta}
\eeq
\section{\label{NRRBchecks}Some checks on our code}
\subsection{Nonrotating, noninteracting, nonrelativistic, finite chemical potential in 1, 2, and 3 dimensions}
The lattice action for a nonrotating, noninteracting, and nonrelativistic system is the following:
\beq
S_{\text{lat},r} = \phi_{r}^{*}\left[\phi_{r} - e^{d\tau\mu} \phi_{r - \hat{\tau}} - \frac{d\tau}{2m}\sum_{i =1}^{d}\left(  \phi_{r + \hat{i}} - 2 \phi_{r} + \phi_{r - \hat{i}}\right) \right].
\eeq
This can be written as fields multiplying a matrix:
\beq
S_{\text{lat},r} = \sum_{r}\sum_{r'}\phi_{r}^{*}M\phi_{r'} = \sum_{r}\sum_{r'}\phi_{r}^{*}\left[ (1 + \frac{d\tau d}{m})\delta_{r,r'}  - e^{d\tau\mu} \delta_{r - \hat{\tau},r'}- \frac{d\tau}{2m}\sum_{i =1}^{d}\left(\delta_{r+\hat{i},r'} + \delta_{r-\hat{i},r'} \right)\right]\phi_{r'},
\eeq
which we can use to determine analytically the density and field modulus squared of this system in order to check against our code's results. Recall that
\bea
\left \langle \hat{n} \right \rangle & = & \frac{-1}{V}\frac{\partial ln \CZ}{\partial (\beta \mu)} = \frac{-1}{V}\frac{\partial}{\partial (\beta\mu)} (-ln(det(M))) \\
& = & \frac{1}{V}\frac{\partial}{\partial (\beta\mu)}Tr(ln M) = \frac{1}{V}\frac{\partial}{\partial (\beta\mu)}\sum_{k}ln D_{kk} = \frac{1}{V}\sum_{k}\frac{1}{D_{kk}}\frac{\partial D_{kk}}{\partial (\beta\mu)}\nonumber
\eea with $\beta \mu = N_{\tau} d\tau \mu$, and note that for a nonrelativistic system,
%
\bea
\left \langle \phi^{*}\phi \right \rangle & =& \frac{-1}{V}\frac{\partial ln \CZ}{\partial  (d/m)} = \frac{-1}{V}\frac{\partial}{\partial  (d/m)} (-ln(det(M))) \\
& =& \frac{1}{V}\frac{\partial}{\partial (d/m)}Tr(ln M) = \frac{1}{V}\frac{\partial}{\partial  (d/m)}\sum_{k}ln D_{kk} \nonumber \\
&= &\frac{1}{V}\sum_{k}\frac{1}{D_{kk}}\frac{\partial D_{kk}}{\partial  (d/m)} = \sum_{k}\frac{1}{D_{kk}}. \nonumber
\eea
%
\subsubsection{Diagonalizing our matrix, M}
We can represent the nonrotating, noninteracting action as
\beq
S[\lambda = \omega = 0] = \sum_{r,r'}\phi^{*}_{r}M_{r,r'}[d,m,\mu]\phi_{r'}
\eeq
where 
\beq
M_{r,r'}[d,m,\mu] = \left[ (1+\frac{d\tau d}{m})\delta_{r,r'} - e^{d\tau\mu}\delta_{r-\hat{t},r'}-\frac{d\tau}{2m}\sum_{i=1}^{d}(\delta_{r+\hat{i},r'}+\delta_{r-\hat{i},r'})\right].
\eeq 
We want to diagonalize $M$ by applying a transformation matrix, such that $D_{kk'} = U^{\dagger}MU$, where
\bea
U_{r,k}& =& \frac{\sqrt{2^{d}}}{\sqrt{N_{x}^{d} N_{\tau}}}e^{i k_{0} t}\prod_{i=1}^{d}\sin(k_{i}x_{i}) \\
U^{\dagger}_{r,k}& =& \frac{\sqrt{2^{d}}}{\sqrt{N_{x}^{d} N_{\tau}}}e^{-i k_{0} t}\prod_{i=1}^{d}\sin(k_{i}x_{i}) \\
k_{0} & = & \frac{2 \pi n_{0}}{N_{\tau}},\ n_{0} \in [1,2,...,N_{\tau}] \\
k_{i} & = & \frac{\pi n_{i}}{(N_{x}+1)},\ n_{i} \in [1,2,...,N_{x}].
\eea
Applying the transformation matrix, we get
\bea
D_{k,k'} &=& \frac{2^{d}}{N_{x}^{d} N_{t}}\sum_{r,r'}e^{-i k_{0} t}\prod_{i=1}^{d}\sin(k_{i}x_{i})\left[ (1+\frac{d\tau d }{m})\delta_{r,r'} \right]e^{i k'_{0} t'}\prod_{i=1}^{d}\sin(k'_{i}x'_{i})\nonumber \\
&& -\frac{2^{d}}{N_{x}^{d} N_{t}}\sum_{r,r'}e^{-i k_{0} t}\prod_{i=1}^{d}\sin(k_{i}x_{i})\left[e^{ d\tau \mu}\delta_{r-\hat{t},r'}\right]e^{i k'_{0} t'}\prod_{i=1}^{d}\sin(k'_{i}x'_{i})  \\
&& -\frac{2^{d}}{N_{x}^{d} N_{t}}\sum_{r,r'}e^{-i k_{0} t}\prod_{i=1}^{d}\sin(k_{i}x_{i})\left[\frac{d\tau}{2m} \sum_{i=1}^{d}\delta_{r+\hat{i},r'}\right]e^{i k'_{0} t'}\prod_{i=1}^{d}\sin(k'_{i}x'_{i})\nonumber\\
&& -\frac{2^{d}}{N_{x}^{d} N_{t}}\sum_{r,r'}e^{-i k_{0} t}\prod_{i=1}^{d}\sin(k_{i}x_{i})\left[\frac{d\tau}{2m}\sum_{i=1}^{d}\delta_{r-\hat{i},r'}\right]e^{i k'_{0} t'}\prod_{i=1}^{d}\sin(k'_{i}x'_{i})\nonumber.
\eea
Resolving the delta functions, performing the sum over $r'$, and pulling everything that does not depend on $r = (t,\vec{x})$ outside the sum, this reduces to
\bea
D_{k,k'} &=& \frac{2^{d}}{N_{x}^{d} N_{t}} (1+\frac{d\tau d}{m})\sum_{r}e^{-i t(k_{0} -k'_{0})}\prod_{i=1}^{d}\sin(k_{i}x_{i})\sin(k'_{i}x_{i})\nonumber \\
&& -\frac{2^{d}}{N_{x}^{d} N_{t}}e^{d\tau\mu}e^{-ik'_{0}}\sum_{r}e^{-i t(k_{0}- k'_{0})}\prod_{i=1}^{d}\sin(k_{i}x_{i})\sin(k'_{i}x_{i})  \\
&& -\frac{2^{d}}{N_{x}^{d} N_{t}}\frac{d\tau}{2m}\sum_{r}e^{-i t(k_{0} -k'_{0})}\prod_{i=1}^{d}\sin(k_{i}x_{i})\sum_{i=1}^{d}\sin(k'_{i}x_{i}+k'_{i})\nonumber\\
&& -\frac{2^{d}}{N_{x}^{d} N_{t}}\frac{d\tau}{2m}\sum_{r}e^{-i t(k_{0} -k'_{0})}\prod_{i=1}^{d}\sin(k_{i}x_{i}) \sum_{i=1}^{d}\sin(k'_{i}x_{i}-k'_{i})\nonumber.
\eea
To further expand the last two lines, we use the following trig identity: $\sin(a \pm b) = \sin(a)\cos(b) \pm \sin(b)\cos(a)$, which gives us:
\bea
D_{k,k'} &=& \frac{2^{d}}{N_{x}^{d} N_{t}} (1+\frac{ d\tau d}{m} -e^{d\tau\mu}e^{-ik'_{0}} )\sum_{r}e^{-i t(k_{0} -k'_{0})}\prod_{i=1}^{d}\sin(k_{i}x_{i})\sin(k'_{i}x_{i}) \\
&& -\frac{2^{d}}{N_{x}^{d} N_{t}}\frac{d\tau}{2m}\sum_{i=1}^{d}\cos(k'_{i})\sum_{r}e^{-i t(k_{0} -k'_{0})}\prod_{i=1}^{d}\sin(k_{i}x_{i})\sin(k'_{i}x_{i})\nonumber\\
&& -\frac{2^{d}}{N_{x}^{d} N_{t}}\frac{d\tau}{2m}\sum_{i=1}^{d}\sin(k'_{i})\sum_{r}e^{-i t(k_{0} -k'_{0})}\prod_{i=1}^{d}\sin(k_{i}x_{i})\cos(k'_{i}x_{i})\nonumber\\
&& -\frac{2^{d}}{N_{x}^{d} N_{t}}\frac{d\tau}{2m}\sum_{i=1}^{d}\cos(k'_{i})\sum_{r}e^{-i t(k_{0} -k'_{0})}\prod_{i=1}^{d}\sin(k_{i}x_{i}) \sin(k'_{i}x_{i})\nonumber\\
&& +\frac{2^{d}}{N_{x}^{d} N_{t}}\frac{d\tau}{2m} \sum_{i=1}^{d}\sin(k'_{i})\sum_{r}e^{-i t(k_{0} -k'_{0})}\prod_{i=1}^{d}\sin(k_{i}x_{i})\cos(k'_{i}x_{i})\nonumber.
\eea
Using the following Fourier identities
\bea
\sum_{x}\sin(kx)\sin(k'x) & = &\frac{N_{x}}{2}\delta_{k,k'}\nonumber\\
\sum_{x}\sin(kx)\cos(k'x) & = &0 \\
\sum_{x}e^{-i x (k-k')} &=& N_{x}\delta_{k,k'}\nonumber\\
\eea we find that
\bea
D_{k,k'} &=& \frac{2^{d}}{N_{x}^{d} N_{t}} \left(1+\frac{d\tau d}{m} -e^{d\tau\mu}e^{-ik'_{0}} -\frac{d\tau}{m}\sum_{i=1}^{d}\cos(k'_{i})\right)N_{t}\delta_{k_{0},k'_{0}}\prod_{i=1}^{d}\frac{N_{x_{i}}}{2} \delta_{k_{i},k'_{i}}\nonumber\\
&=& \frac{2^{d}}{N_{x}^{d} N_{t}} \left(1+\frac{ d\tau d}{m} -e^{ d\tau\mu}e^{-ik'_{0}} -\frac{d\tau}{m}\sum_{i=1}^{d}\cos(k'_{i})\right)N_{t}\left(\frac{N_{x}}{2}\right)^{d} \delta_{k,k'}\nonumber\\
D_{k,k'} &=& \left(1+\frac{d\tau d}{m} -e^{d\tau\mu}e^{-ik'_{0}} -\frac{d\tau}{m}\sum_{i=1}^{d}\cos(k'_{i})\right) \delta_{k,k'}\nonumber,
\eea
or, slightly rearranged:
\beq
D_{k,k'} = \left(1 -e^{ d\tau\mu}e^{-ik'_{0}} +\frac{d\tau}{m}\sum_{i=1}^{d}(1-\cos(k'_{i}))\right) \delta_{k,k'}.
\eeq
Note that this is a complex matrix, with real and imaginary parts:
\bea
\text{Re}\left[D_{k,k'}\right] &=&  \left(1 -e^{ d\tau\mu}\cos(k'_{0}) +\frac{d\tau}{m}\sum_{i=1}^{d}(1-\cos(k'_{i}))\right) \delta_{k,k'} \\
\text{Im}\left[D_{k,k'}\right] &=&  e^{ d\tau\mu}\sin(k'_{0}) \delta_{k,k'} .
\eea

\subsubsection{Analytical solution for the nonrotating, noninteracting density}
We can now use our diagonal matrix $D_{k,k'} = D_{k,k}$ to solve for the density of this system. Recall that
\beq
\left \langle \hat{n} \right \rangle = \frac{1}{V N_{\tau}}\sum_{k}\frac{1}{D_{kk}}\frac{\partial D_{kk}}{\partial (d\tau\mu)}.
\eeq
We first need to solve for $\frac{\partial D_{kk}}{\partial (d\tau\mu)}$:
\bea
\frac{\partial D_{kk}}{\partial \mu} & = & \frac{\partial}{\partial (d\tau\mu)}\left[ \left(1 -e^{d\tau\mu}e^{-ik'_{0}} +\frac{d\tau}{m}\sum_{i=1}^{d}(1-\cos(k'_{i}))\right) \delta_{k,k'}\right] \\ \nonumber
& = & - e^{d\tau\mu}e^{-ik'_{0}}\delta_{k,k'}.
\eea Plugging this in to our equation for the density gives us:
\beq
\left \langle \hat{n} \right \rangle = \frac{1}{N_{x}^{d}N_{t}}\sum_{k}\frac{D^{*}_{kk}}{|D_{kk}|^{2}}\left( -e^{d\tau\mu}e^{-ik'_{0}}\delta_{k,k'}\right).
\eeq

\subsubsection{Analytical solution for the nonrotating, noninteracting field modulus squared}
\beq
\left \langle \phi^{*}\phi \right \rangle  = \sum_{k}\frac{1}{D_{kk}} = \sum_{k}\frac{D^{*}_{kk}}{|D_{kk}|^{2}}.
\eeq


\subsubsection{Analytical solution for the two-point correlation function}
The eigenvalues and eigenvectors can be used to compute the solution for two-point correlation function for this system:
\beq
G(x, x') = \sum_{k} \frac{1}{D_{kk}}U^{\dagger}_{k}(x)U_{k}(x')
\eeq

\end{document}